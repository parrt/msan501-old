\documentclass[titlepage]{tufte-book}

\usepackage[runall=true]{pythontex}
\usepackage{amsthm}
\usepackage{amsmath}
\usepackage{amssymb}
\usepackage[pdftex]{graphicx}
\usepackage{epstopdf}
\usepackage{hyperref}
\usepackage{alltt}
\usepackage{listings}
\usepackage{array}
\usepackage{extarrows}
\usepackage{setspace}
\usepackage{tikz}
\usepackage{tikz-qtree}
\usetikzlibrary{calc}
\usetikzlibrary{positioning}
\usepackage{hyperref}
\usepackage{graphviz}
\usepackage{geometry}                % See geometry.pdf to learn the layout options. There are lots.
\usepackage{bashful}
\usepackage{microtype} % Improves character and word spacing

\usepackage{lipsum} % Inserts dummy text

\usepackage{booktabs} % Better horizontal rules in tables

\setkeys{Gin}{width=\linewidth,totalheight=\textheight,keepaspectratio} % Improves figure scaling
\graphicspath{{figures/}}

\usepackage{fancyvrb} % Allows customization of verbatim environments
\fvset{fontsize=\normalsize} % The font size of all verbatim text can be changed here

\newcounter{problem}
\newcounter{total}
\newcommand{\step}[1]{{}
\vspace{4pt} \noindent {\bf \theproblem. }#1\addtocounter{problem}{1}}

\newcommand{\cut}[1]{}

\newcommand{\chili}{\scalebox{.04}{\includegraphics{figures/chili.pdf}}}
\newcommand{\chchili}{{\chili\chili}}
\newcommand{\chchchili}{{\chchili\chili}}

\hypersetup{
urlcolor=blue,
colorlinks=true
}
\usepackage[noline, procnumbered, linesnumberedhidden, boxed]{algorithm2e}

\newcommand{\openepigraph}[2]{ % This block sets up a command for printing an epigraph with 2 arguments - the quote and the author
\begin{fullwidth}
\sffamily\large
\begin{doublespace}
\noindent\allcaps{#1}\\ % The quote
\noindent\allcaps{#2} % The author
\end{doublespace}
\end{fullwidth}
}

\newcommand{\blankpage}{\newpage\hbox{}\thispagestyle{empty}\newpage} % Command to insert a blank page

\usepackage{makeidx} % Used to generate the index
\makeindex % Generate the index which is printed at the end of the document

\renewcommand{\maketitlepage}[0]{%
  \cleardoublepage%
  {%
  \sffamily%
  \begin{fullwidth}%
  ~
  \vspace{11.5pc}%
  \fontsize{36}{40}\selectfont\par\noindent\textcolor{darkgray}{\allcaps{\thanklesstitle}}%
  
\scalebox{.2}{\includegraphics{figures/msan-logo}}
  \vspace{11.5pc}%
  \fontsize{12}{18}\selectfont\par\indent\textcolor{darkgray}{\allcaps{\thanklessauthor}\\
\indent{\tt parrt@cs.usfca.edu}\\
\href{http://parrt.cs.usfca.edu}{http://parrt.cs.usfca.edu}}%
  \vspace{11.5pc}%
  \fontsize{14}{16}\selectfont\par\noindent\allcaps{\thanklesspublisher}%
  \end{fullwidth}%
  }
  \thispagestyle{empty}%
  \clearpage%
}

\titlecontents{part}% FIXME
    [0em] % distance from left margin
    {\vspace{1.5\baselineskip}\begin{fullwidth}\LARGE\rmfamily\itshape} % above (global formatting of entry)
    {\contentslabel{2em}} % before w/label (label = ``II'')
    {} % before w/o label
    {\rmfamily\upshape\qquad\thecontentspage} % filler + page (leaders and page num)
    [\end{fullwidth}] % after

  \titlecontents{chapter}%
    [0em] % distance from left margin
    {\vspace{1.5\baselineskip}\begin{fullwidth}\Large\rmfamily\itshape} % above (global formatting of entry)
    {\hspace*{0em}\contentslabel{2em}} % before w/label (label = ``2'')
    {\hspace*{4em}} % before w/o label
    {\rmfamily\upshape\qquad\thecontentspage} % filler + page (leaders and page num)
    [\end{fullwidth}] % after

\titlespacing*{\chapter}{0pt}{0pt}{30pt}
\titlespacing*{\section}{0pt}{3.5ex plus 1ex minus .2ex}{2.3ex plus .2ex}
\titlespacing*{\subsection}{0pt}{3.25ex plus 1ex minus .2ex}{1.5ex plus.2ex}

\title{
Exercises in\\
Computational\\
Analytics
}

% preparatory, introductory, foundations, preliminary, survey
\author{Terence Parr}

\date{} % delete this line to display the current date

%\setcounter{secnumdepth}{0}

\begin{document}

\frontmatter

%----------------------------------------------------------------------------------------

\maketitle % Print the title page

%----------------------------------------------------------------------------------------
%	COPYRIGHT PAGE
%----------------------------------------------------------------------------------------

\newpage
\begin{fullwidth}
~\vfill
\thispagestyle{empty}
\setlength{\parindent}{0pt}
\setlength{\parskip}{\baselineskip}
Copyright \copyright\ \the\year\ Terence Parr

\par\smallcaps{A Bonkers the Cat Production}

\vspace{-65pt}
\scalebox{.35}{\includegraphics{figures/bonkers}}

\end{fullwidth}

%----------------------------------------------------------------------------------------

\tableofcontents % Print the table of contents

\mainmatter

\chapter*{Introduction}

\marginnote[0.31in]{You will work mostly on your own laptops, but you must get familiar with the UNIX command line.  It's also important to learn how to install software and execute commands on a remote server; servers or what provide the websites you visit while browsing and they provide services to mobile apps on your phone. We've received an educational grant from Amazon to use their compute cloud called Amazon Web Services (AWS). We also have access to an IBM cluster, housed in the College of arts and sciences at USF.}

\newthought{Welcome to MSAN501}, the computational analytics boot camp at the University of San Francisco! This exercise book collects all of the labs you must complete by the end of the boot camp in order to pass.  The labs start out as very simple tasks or step-by-step recipes but then accelerate in difficulty, culminating with an interesting text analysis project. You will build all projects with Python (version 2, not 3).

This course is specifically designed as an introduction to analytics programming for those who are not yet skilled programmers. The course also explores many concepts from math and statistics, but in an empirical fashion rather than symbolically as one would do in a math class. Consequently, this course is also useful to programmers who would like to strengthen their understanding of numerical methods.

\marginnote[.2in]{
\scalebox{.55}{\includegraphics{figures/clt_unif-2000-20.pdf}}
\scalebox{.55}{\includegraphics{figures/conf-500.pdf}}
\scalebox{.55}{\includegraphics{figures/wage-murders-cost-3d.pdf}}
\scalebox{.55}{\includegraphics{figures/wage-murders-heatmap-trace1.pdf}}
}

The exercises are grouped into four parts. We begin with simple programs to compute statistics, build simple data structures, and use libraries to create visualizations. The second part strives to give an intuitive feel for random variables, density functions, the central limit theorem, hypothesis testing, and confidence intervals. It's one thing to learn about their formal definitions, but to get a really solid grasp of these concepts, it really helps to observe statistics in action. All of the techniques we'll use in empirical statistics rely on the ability to generate random values from a particular distribution. We can do it all from a uniform random number generator, which is the first exercise in that part.

The third set of exercises deals with function optimization. Given a particular function, $f(x)$, optimizing it generally means finding its minimum or maximum, which occur when the derivative goes flat: $f'(x) = 0$. When the function's derivative cannot be derived symbolically, we're left with a general technique called {\em gradient descent} that searches for minima. It's like putting a marble on a hilly surface and letting gravity bring it to the nearest minimum.

\marginnote{As you progress through these exercises, you'll learn a great deal about Python and the following libraries: {\tt matplotlib}, {\tt numpy}, {\tt scipy}, and {\tt py.test}.   I also recommend that you learn how to use a Python development environment called \href{http://www.jetbrains.com/pycharm/}{PyCharm}, for which we have been granted a site license.}

Finally, part four has an exercise that introduces text analysis. We will compute something called {\em TFIDF} that indicates how well that word distinguishes a document from other documents in a corpus.  That score is used broadly in text analytics, but our exercise uses it to summarize documents by listing the most important words.

\part{Python Programming and Data Structures}
 
\chapter{Computing Point Statistics}

\setcounter{problem}{1}
\section{Discussion}

\begin{fullwidth}

The goal of this task is to refresh your memory of a few point statistics. 

\subsection{Stats}

This exercise involves writing functions to compute sample mean, variance, and covariance from a data set (list of values).  In mathematics notation, the sample estimates are:

\[\tag{Sample mean}
\bar x = \frac{1}{N} \sum_{i=1}^{N} x_i 
\]

\[\tag{Unbiased sample variance}
s^2 = \frac{1}{N-1} \sum_{i=1}^{N} (x_i - m)^2
\]

\[\tag{Unbiased sample covariance}
cov(x,y) = \frac{1}{N-1} \sum_{i=1}^{N} (x_i - \bar x)(y_i - \bar y)
\]

In Python, you must define functions {\tt mean(x)}, {\tt var(x)}, {\tt cov(x,y)} where {\tt x} and {\tt y} are objects that behave like a list or iterator. The functions return a floating-point value based on the above mathematics notation. If the length of the incoming vectors to {\tt cov} are not the same, return 0. 

\subsection{Libraries}

While we're at it let's learn about creating and importing our own libraries.  You'll notice that {\tt test\_point\_stats.py} references your code like this:

\begin{pyverbatim}
from stats import *
\end{pyverbatim}

\noindent That lets us directly access the functions defined in the {\tt stats.py} file you are going to create.
 
You can test the correctness of the functions by using the {\tt numpy} lib, make sure you ask for the sample population statistics by using parameter {\tt ddof=1} for {\tt var()} and {\tt cov()}. E.g., {\tt np.var(data, ddof=1)}. Be careful not to confuse function names; e.g., {\tt numpy} has functions with the same names (although cov() returns a covariance matrix).

\begin{pyverbatim}
import numpy as np  # np is an alias for the numpy library
x = ...
y = ...
print np.cov(x,y)[0][1] # np.cov returns cov matrix
\end{pyverbatim}

We now have the kernel of a small statistics library in {\tt stats.py} and we will continue to add functions to this as we go along. 

\subsection{Testing}

In computer science, programmers recognize two primary kinds of tests: {\em unit tests} and {\em functional tests}. A unit test is really just testing a function or a few functions whereas functional tests test the overall functionality of the program. In file {\tt test\_point\_stats.py}, I have provided a set of unit tests that you can use for basic sanity checking of your project. 

To make the unit tests work, make sure that you install \href{http://pytest.org/latest/getting-started.html}{py.test}, which is usually just a matter of:

\begin{alltt}
easy_install -U pytest
\end{alltt}

Test your code using the following command line (with your {\tt stats.py} is in the same directory):

\begin{lstlisting}[style=BashInputStyle]
$ python -m pytest test_point_stats.py
============================= test session starts ==============================
platform darwin -- Python 2.7.6 -- py-1.4.30 -- pytest-2.7.2
rootdir: /Users/parrt/msan501/stats, inifile: 
collected 3 items 

test_point_stats.py ...
=========================== 3 passed in 0.01 seconds ===========================
\end{lstlisting}

\noindent If you don't see all tests passing, and there is a problem at a basic level with your software.

\begin{callout}{\bctakecare}
You may not use sum() or any other built-in functions for this project to compute the point statistics.  The whole point of the exercise is to learn to build your own for loops. Obviously.
\end{callout}

\begin{callout}{\bcplume}
{\bf Deliverables}. Make sure that {\tt stats/stats.py} (the functions inside should emit no output at all, just return data as specified) is correctly committed to your repository and pushed to github. 
\end{callout}


\end{fullwidth}


\chapter{Approximating $\sqrt{n}$ with the Babylonian Method}

\setcounter{problem}{1}
\section{Motivation}

\begin{fullwidth}

This lab is really a fancy way to learn about looping in Python and how to quickly prototype something in Excel (if warranted). It also gets you used to encoding mathematical expressions and recurrence relations in Python.
 
\section{Discussion}

To approximate square root, $\sqrt{n}$, the idea is to pick an initial estimate, $x_0$, and then iterate with better and better estimates, $x_{i}$, using the recurrence relation:

\[
x_{i+1} = \frac{1}{2} (x_i + \frac{n}{x_i})
\]

To see how this works, jump into Excel (yes, a spreadsheet) and crank through a few iterations by defining cells with $n$ and your initial estimate $x_0$, which can be anything you want. (It's sometimes easier to play around without having to deal with a programming language.) Then you need to define a cell that computes the above better approximation using $x_i$ as the cell above it. I hardcoded the names in column A and the first two rows of column B. Cell B3 should be a formula that computes B4 based upon B3. Then you can extend the formula down and watch it converge on the final (correct) value for $\sqrt{125348}$. My spreadsheet looks like this:

~\\
\scalebox{.25}{\includegraphics{figures/sqrt-excel.pdf}}
~\\

Try out any nonnegative number and you'll see that it still converges, though at different rates.

There's a great deal on the web you can read to learn more about why this process works but it relies on the average (midpoint) of $x$ and $n/x$ getting us closer to $\sqrt{n}$.  It can be shown that if $x$ is above $\sqrt{n}$ then $n/x$ is below $\sqrt{n}$ and the reverse is true if $x$ is below the root.  The iteration converges and does so quickly. Informally, as shown in Wikipedia, we can represent the true square root by adding an error term to our estimate:

\[
\sqrt{n} = x + \epsilon
\]

or,

\[
n = (x + \epsilon)^2
\]

Expanding, we get:

\[
n = x^2 + 2x\epsilon + \epsilon^2
\]

Solving for $\epsilon$:

\[
n - x^2 = \epsilon (2x + \epsilon)
\]

\[
\epsilon = \frac{n - x^2}{2x + \epsilon}
\]

Because $\epsilon$ is much smaller than $x$, we can drop it from the denominator leaving us with an estimate of epsilon:

\[
\epsilon = \frac{n - x^2}{2x}
\]

Then we can plug it back into $x + \epsilon$ and get:

\[
x := x + \epsilon = x + \frac{n - x^2}{2x} = \frac{2x^2}{2x} + \frac{n - x^2}{2x} = \frac{1}{2}\frac{x^2 + n}{x} = \frac{1}{2}(x + \frac{n}{x})
\]

Which gets its back to the Babylonian formula. Since we dropped an $\epsilon$ term, this formula for $x$ is inexact but it gets us closer to $\sqrt{n}$.

\cut{
To play first use Excel: 
=0.5*(B2+\$B\$1/B2)
}

Now that you understand how this estimate works, your goal is to implement a simple Python method called sqrt() that uses the Babylonian method to approximate the square root. Here is a starter kit for you. Please call the file {\tt sqrt.py}.

\begin{pyverbatim}
import math

# Stop iterating when the new approximation is within
# PRECISION of the old value.
PRECISION = 0.000001

# compute square root of n
def sqrt(n):
    x_0 = 1.0 # pick any old value
    ... fill this in ...

def check(n):
    delta = math.sqrt(n) - sqrt(n)
    if math.abs(delta) > PRECISION:
        raise BaseException("Inaccurate sqrt(%f)=%f; estimate is %f" %
            (n, math.sqrt(n), sqrt(n)))

# check a range of values
check(125348)
check(100)
check(1)
check(0)
\end{pyverbatim}

\section{Deliverables}

Please submit:

\begin{itemize}
\item a PDF showing a snapshot of your spreadsheet
\item the formula you used in B3 {\bf and} B4.
\item your {\tt sqrt.py} Python file
\end{itemize}

{\em You may not use math.sqrt() for implementing your function, but you may use it for testing the results.  Obviously.}

\end{fullwidth}

\chapter{Generating Uniform Random Numbers}

\setcounter{problem}{1}

\begin{quote}
{\bf Q}: {\em How to generate pure random string}? \\
{\bf A}: {\em Put a fresh student in front of vi editor and ask him to quit}.\\
~~~~--- Emiliano Lourbet (@taitooz)
\end{quote}

\section{Discussion}

\begin{fullwidth}

To perform computer-based simulation we need to be able to generate random numbers. Generating random numbers following a uniform distribution are the easiest to generate and are what comes out of the standard programming language ``give me a random number'' function.  Here's a sample Python session:

\begin{pyconsole}
import random
print random.random()
print random.random()
print random.random()
\end{pyconsole}

We could generate real random numbers by accessing, for example, noise on the ethernet network device but that noise might not be uniformly distributed. We typically generate pseudorandom numbers that aren't really random but look like they are. From Ross' {\em Simulation},  we see a very easy recursive mechanism that generators values in $[0,m-1]$:\\

$x_n = a x_{n-1}$ modulo $m$\\

That's recursive (or iterative and not {\em closed form}) because $x_n$ is a function of a prior value: \\

$x_1 = ax_0$ modulo $m$\\
$x_2 = ax_1$ modulo $m$\\
$x_3 = ax_2$ modulo $m$\\
$x_4 = ax_3$ modulo $m$\\
$...$\\

\noindent To get random numbers between 0 and 1, we use $x_n / m$.

We must pick a value for $a$ and $m$ that make $x_n$ seem random. Ross suggests choosing a large prime number for $m$ that fits in our integer word size, e.g., $m = 2^{31} - 1$, and $a = 7^5 = 16807$.

Initially we set a value for $x_0$, called the {\em random seed} (it is not the first random number). Every seed leads to a different sequence of pseudorandom numbers. In Python, you can set the seed of the standard library by using {\tt random.seed([x])}.

Your goal is to take that simple recursive formula and use it to generate the first 10 random numbers using a {\tt for} loop in Python as part of the ``main'' code. Please use file {\tt varunif.py} and make function {\tt runif()} that returns a new random value per call. Use $m = 2^{31} - 1$, $a = 7^5 = 16807$, and an initial seed of $x_0 = 666$.  Your output should look something like:

\begin{verbatim}
0.00521236192678
0.604166903349
0.233144581892
0.460987861017
0.822980116505
0.826818094508
0.331714398848
0.1239014343
0.411406287184
0.505468696591
\end{verbatim}

Because we are all using the same seed, the sequence of numbers should be the same.

Next, make function {\tt runif\_(a,b)} that returns random values between {\tt a} and {\tt b}. Your function definitions should look like:

\begin{pyverbatim}
# U(0,1)
def runif():
    ...

# U(a,b)
def runif_(a,b):
    ...
\end{pyverbatim}

\section{Deliverables}

You must submit your {\tt varunif.py} containing a main loop and your functions runif and runif\_. Also submit your output via Canvas as a text file.

\end{fullwidth}

\chapter{Histograms Using {\tt matplotlib}}

\setcounter{problem}{1}
\section{Discussion}

\begin{fullwidth}


The goal of this lab is to teach you the basics of using matplotlib to display probability mass functions, otherwise known as histograms. In this lab we will use the uniform distribution. Use filename {\tt hist.py}.

\section{Steps}

\step  import the proper libraries

\begin{pyverbatim}
import matplotlib.pyplot as plt
import numpy as np
\end{pyverbatim}

\step Get a sample of uniform random variables in $U(0,1)$

\begin{pyverbatim}
N = 1000
X = [np.random.uniform(0,1) for i in range(N)] # U(0,1)
# or np.random.uniform(0,1,N)
\end{pyverbatim}

\step Display a histogram using matplotlib (in a separate window)

\begin{pyverbatim}
plt.hist(X, normed=1)
plt.show()
\end{pyverbatim}

\step Run it. \\

\step Now, save the image as a PDF to the same directory by inserting a save command in between the histogram and the show method:

\begin{pyverbatim}
plt.hist(X, normed=1)
plt.savefig('unif-0-1-density.pdf', format="pdf")
plt.show()
\end{pyverbatim}

\step Run it. Your pdf file should look like

\scalebox{.25}{\includegraphics{figures/unif-0-1-density.pdf}}

\step  Graphs should always have the axes labeled. Let's do that as well as add a title and set the range of the graph. Put this code right before the {\tt savefig()}.

\begin{pyverbatim}
plt.title("U(0,1) Density Demo")
plt.xlabel("X", fontsize=16)
plt.ylabel("Density", fontsize=16)
plt.axis([0, 1, 0, 2])
\end{pyverbatim}

\step Run it. \\

\step It's also common to add some annotations inside the graph to explain more about what we are seeing. First, we need to get access to the figure itself and then has to figure about its axes. (We need this in order to specify coordinates within the graph.) Put the following code before the {\tt hist()} call.

\begin{pyverbatim}
fig = plt.figure()        # get a handle on the figure object itself
ax = fig.add_subplot(111) # weird stuff to get the Axes object within figure
\end{pyverbatim}

Then, before the {\tt savefig()},  add the following to display some text above the histogram within the graph. The coordinates are from 0..1 where 0 is the left/bottom edge and 1 is the right/top edge.

\begin{pyverbatim}
# put N=... at top left
plt.text(.1,.9, 'N = %d' % N,
		 fontsize=16,
		 transform = ax.transAxes) 
\end{pyverbatim}

\step Also, let's change the file name slightly so we can keep our original graph plus our fancy one:

\begin{pyverbatim}
plt.savefig('unif-0-1-density-fancy.pdf', format="pdf")
\end{pyverbatim}

\step Run it. \\

\scalebox{.25}{\includegraphics{figures/unif-0-1-density-fancy.pdf}}

{\bf To understand distributions, it's a great idea to start messing around with the parameters of the density or mass function.} \\

\step Change $U(0,1)$ to $U(2,8)$ and examine the results. You will have to alter the axis() to use different ranges. (Or let the plotting software do the work for you and get rid of the axis() call.) Run it. You should see something like the following.

\scalebox{.25}{\includegraphics{figures/unif-2-8-density-fancy.pdf}}

\section{Deliverables}

Please submit:

\begin{itemize}
\item your {\tt hist.py} Python file
\item a PDF of your $U(2,8)$ graph.
\end{itemize}

\end{fullwidth}

\chapter[AWS]{Lab 8: Launching a virtual machine at Amazon Web Services}
\label{ch:1}

\setcounter{problem}{1}

\section{Discussion}

\begin{fullwidth}

The goal of this lab is to teach you to create a Linux machine, login, and copy some data to that machine.

\section{Steps}

\step Go to your EC2 dashboard at AWS and click "Launch Instance"; Use the classic Wizard.

\step Select the ``Amazon Linux AMI'' server, which should be the first one. 64bit.

\step Select instance type ``m1.small,'' which should be the second machine type listed. then click continue.

\step Skip over the advanced instance options by clicking continue.

\step Skip over the storage device configuration

\step For the key named "Name", change the value to something like {\em youruserID}-windows or something like that so that you can identify it later if you have multiple machines going. click continue.

\step The first time, you will need to create a new key pair. Name it as your user ID then click on the create and download keyfile. It will leave you with a {\em userid}.pem file, which are your security credentials for getting into the machine. Save that file in a safe spot. If you lose it you will not be able to get into your machine that you create.

\step Choose the default security group, which controls the firewall.

\step Tell it to create the instance. Close that pop up and it will take you to the instances you.

\step We need port 22 open so that {\tt ssh} can get through the firewall to our computer. ssh listens at port 22 for connections from the outside world. Click on security groups in the left gutter. Click on ``default'' and then select the ``inbound'' tab. Put 22 in the port range and click Add Rule. Then click apply rule changes. You will see now that it allows port 22 (SSH) connections.

\step Go back to the instances view by clicking instances in the left gutter. Right-click on your instance in the display and select Connect. Click on the ``Connect with a standalone SSH client'' link and then inside you will see the {\tt ssh} command necessary to connect to your machine. If you have Windows, there is a link to show you how to use an SSH client called PuTTY. For mac and linux users, we will use the direct ssh command from the command line. It will be something like:

{\small
\begin{alltt}
ssh -i parrt.pem ec2-user@ec2-23-22-115-148.compute-1.amazonaws.com
\end{alltt}
}

\noindent Naturally, you will have to provide the full pathname to your user.pem file.

\step  Before we can connect, we have to make sure that the security file is not visible to everyone on the computer (other users). Otherwise ssh will not let us connect because the security file is not secure.

{\small
\begin{alltt}$ cd ~/Dropbox/Terence
$ ls -l parrt.pem
-rw-r--r--@ 1 parrt  parrt  1696 Aug  4 15:15 /Users/parrt/Dropbox/Terence/parrt.pem
\end{alltt}
}


\noindent To fixed the permissions, we can use whatever ``show information about file'' GUI your operating system has or, from the command line, do this:

{\small
\bash[script,stdout,prefix=$]
cd ~/Dropbox/Terence
chmod 600 parrt.pem
\END
}

\noindent which changes the permissions like this:

{\small
\bash[script,stdout,prefix=$ ]
cd ~/Dropbox/Terence
ls -l parrt.pem
\END
}

\noindent Don't worry if you don't understand exactly what's going on there. it's basically saying that the file is only read-write for me, the current user, with no permissions to anybody else. If you don't do this properly, you will see something like this error when you try to connect later:

{\small
\begin{alltt}@@@@@@@@@@@@@@@@@@@@@@@@@@@@@@@@@@@@@@@@@@@@@@@@@@@@@@@@@@@
@         WARNING: UNPROTECTED PRIVATE KEY FILE!          @
@@@@@@@@@@@@@@@@@@@@@@@@@@@@@@@@@@@@@@@@@@@@@@@@@@@@@@@@@@@
Permissions 0644 for '/Users/parrt/Dropbox/Terence/parrt.pem' are too open.
\end{alltt}
}

\step Go to a shell on your mac or linux machine (or PuTTY on windows) and connect to the remote server.  On my mac, it looks like this:

{\small
\begin{alltt}
$ ssh -i ~/Dropbox/Terence/parrt.pem ec2-user@ec2-23-22-115-148.compute-1.amazonaws.com

       __|  __|_  )
       _|  (     /   Amazon Linux AMI
      ___|\___|___|

https://aws.amazon.com/amazon-linux-ami/2013.03-release-notes/
There are 6 security update(s) out of 13 total update(s) available
Run "sudo yum update" to apply all updates.
[ec2-user@ip-10-242-203-15 ~]$ 
\end{alltt}
}

\step To get data up to the server, you can cut-and-paste if the file is small. For example,  cut-and-paste the following data into a file called {\tt coffee} in your home directory. First copy this data from the PDF:

{\small
\begin{alltt}
3 parrt
2 jcoker
8 tombu
\end{alltt}
}

\noindent then type these commands and paste the data in the sequence:

{\small
\begin{alltt}
$ cd ~ # get to my home directory
$ cat > coffee
3 parrt
2 jcoker
8 tombu
^D
$ cat coffee # print it back out
3 parrt
2 jcoker
8 tombu
$ 
\end{alltt}
}

\noindent The {\tt \^{}D} means control-D, which means end of file.  {\tt cat} is reading from standard input and writing to the file. The way it knows we are done is when we signal in the file with control-D.

\step For larger files, we need to use the secure copy {\tt scp} command that has the same argument structure as secure shell {\tt ssh}. From the directory where you have the {\tt coffee} file on your laptop, use the following similar command:

{\small
\begin{alltt}
$ scp -i ~/Dropbox/Terence/parrt.pem access.log \textbackslash
   ec2-user@ec2-23-22-115-148.compute-1.amazonaws.com:~ec2-user
access.log                                    100% 1363KB   1.3MB/s   00:00
$ 
\end{alltt}
}

\noindent Then, from the shell that is connected to the remote server, ask for the directory listing and you will see the new file:

{\small
\begin{alltt}
[ec2-user@ip-10-242-203-15 ~]$ ls
access.log  coffee
\end{alltt}
}


\step STOP YOUR INSTANCE WHEN YOU ARE DONE!  otherwise will continue to get charged for the use of that machine. If you right-click on the instance and say "STOP", it will stop the machine and you do not get charged but you can restart it without having to go through this whole procedure. If you say "TERMINATE", it will toss the machine out and you will have to go through this procedure again.

\section{Deliverables}

None. Please follow along in class.

\end{fullwidth}

\chapter{Graph Adjacency Lists and Matrices}

\setcounter{problem}{1}

\section{Goal}

\begin{fullwidth}

The goal of this task is to teach you about the implementation of graphs in Python, how to implement a few simple related algorithms, and do some simple data loading.  As part of this exercise, you will also learn to transform data, which is an important data preparation skill you will need as an analyst. 

\section{Discussion}

In this project, you will convert a \href{https://raw.githubusercontent.com/parrt/msan501/master/data/graph}{{\textcolor{blue}string representation of a graph}} that looks like this:

\begin{alltt}\small
parrt: tombu, dmose, parrt
tombu: dmose, kg9s
dmose: tombu
kg9s: dmose
\end{alltt}

\noindent to an adjacency list representation and ultimately generate a visual representation via \href{http://www.graphviz.org/}{\textcolor{blue}{graphviz/dot}}:

\begin{center}
\scalebox{.55}{\includegraphics{figures/email-graph.pdf}}
\end{center}

\noindent For fun, you will also create an edge matrix representation:

\[
\bordermatrix{
~ & parrt & tombu & dmose & kg9s \cr
parrt & 1 & 1 & 1 & 0\cr
tombu & 0 & 0 & 1 & 1 \cr
dmose & 0 & 1 & 0 & 0 \cr
kg9s & 0 & 0 & 1 & 0\cr
}
\]

\noindent where the nodes have the following indexes (all we really care about here is the order):
 
\[
\left[
\begin{array}{c}
0: parrt \\
1: tombu \\
2: dmose \\
3: kg9s \\
\end{array}
\right]
\]

The following sections describe the functions you must create in {\tt graph.py}. See a \href{https://github.com/parrt/msan501/blob/master/labs/resources/graph.py}{{\textcolor{blue} {\tt graph.py} starter kit}} I've built for you at github.

\subsection{Processing an adjacency list string}

First, you have to process a string representation of an adjacency list and create an internal data structure:

\begin{pyverbatim}
def adjlist(adj_list):
    """
    Read in adj list and store in form of dict mapping node
    name to list of outgoing edges. Preserve the order you find
    for the nodes.
    """
    ...
\end{pyverbatim}

You will use an ordered dictionary ({\tt OrderedDict}) that maps node name x to a list of target nodes. x will be a string and the target list will be a list of strings. For example, from line in string adj\_list

\begin{alltt}
parrt: tombu, dmose, parrt
\end{alltt}

\noindent you will create an entry in the dictionary with key {\tt parrt} and value:

\begin{pyverbatim}
['tombu', 'dmose', 'parrt']
\end{pyverbatim}

\noindent To process the text, you must split the incoming string into lines and then process them one at a time as each line represents an adjacency list. You will use string functions {\tt split} and (likely) {\tt strip} to process the text. The goal here is to learn how to process text so don't look for built-in functions that do all of this for you.

\noindent Printing the adjacency list dictionary from {\tt adjlist}, we should all get the following output:

\begin{pyverbatim}
OrderedDict([('parrt', ['tombu', 'dmose', 'parrt']),
 ('tombu', ['dmose', 'kg9s']), 
 ('dmose', ['tombu']), 
 ('kg9s', ['dmose'])])
\end{pyverbatim}

\subsection{Adjacency list to adjacency matrix}

Given an adjacency list stored as a dictionary per adjlist(), create a function that converts it to an adjacency matrix:
 
\begin{pyverbatim}
def adjmatrix(adj):
    """
    From an adjacency list, return the adjacency matrix with entries in {0,1}.
    The order of nodes in adj is assumed to be same as they were read in.
    """
    ...
\end{pyverbatim}

\noindent The matrix should look like the one shown above.

\subsection{Getting a list of all nodes}

A very useful function to have is the following that returns a list of all nodes visited starting at a particular node in a graph. 
 
\begin{pyverbatim}
def nodes(adj, start_node):
    """
    Walk every node in graph described by adj list starting at start_node
    using a breadth-first search.  Return a list of all nodes found (in
    any order). Include the start_node.
    """
    ...
\end{pyverbatim}

\noindent Do not build a recursive function as you must do a breadth-first search. (Recursive functions are much more useful when doing a depth-first search.) The basic algorithm looks like this:

\begin{algorithm}[H]
\SetInd{.3em}{.3em}
$visited$ = []\;
add the start node to a work list\;
\While{more work}{
    node = remove a node from work list\;
    add node to $visited$ list\;
    targets = adjacency\_list[node]\;
    add all unvisited targets to work list\;
}
\Return{visited}\;
\end{algorithm}

\subsection{Generating DOT output}

In order to visualize the graph you have read in, create the following function that dumps valid Graphviz DOT code, given an adjacency list. Then cut-and-paste the output and put it into Graphviz to display it.
 
\begin{pyverbatim}
def gendot(adj):
    """
    Return a string representing the graph in Graphviz DOT format
    with all p->q edges. Parameter adj is an adjacency list.
    """
    ...
\end{pyverbatim}

\noindent Or, to amaze your family and friends, you can directly from the command line on a mac or unix box:

\begin{alltt}
python test_dot.py | dot -Tpdf > /tmp/graph.pdf; open /tmp/graph.pdf
\end{alltt}

Here is a simple test rig, {\tt test\_dot.py}, that translates an input string description to DOT and prints it out.

\begin{pyverbatim}
from graph import *

# test dot
g = \
"""
parrt: tombu, dmose, parrt
tombu: dmose, kg9s
dmose: tombu
kg9s: dmose
"""
list = adjlist(g)
dot = gendot(list)
print dot
\end{pyverbatim}

\noindent For the adjacency list shown at the start of this assignment, you should to generate the following DOT code:

\begin{alltt}\small
digraph g \{
  rankdir=LR;
  parrt -> tombu;
  parrt -> dmose;
  parrt -> parrt;
  tombu -> dmose;
  tombu -> kg9s;
  dmose -> tombu;
  kg9s -> dmose;
\}
\end{alltt}

\section{Testing}

I have provided {\tt test\_graph.py} and {\tt test\_dot.py}  test rigs that exercise the required functions using the sample adjacency list described above. Please make sure that your library works with this test rig at minimum.

\section{Deliverables}

Please submit the following via canvas:
 
\begin{itemize}
\item {\tt graph.py} (the functions inside should emit no output at all, just return data as specified)
\item a text file with the output of running {\tt test\_dot.py}, showing the DOT output.
\item a PDF showing the visual representation of the graph as generated by graphviz/dot
\end{itemize}

\end{fullwidth}


\part{Empirical statistics}

\chapter{Generating Binomial Distributions}

\setcounter{problem}{1}
\section{Goal}

\begin{fullwidth}

The goal of this lab is to simulate a binomial distribution using repeated Bernoulli trials and then compare it against the theoretical binomial distribution. Use filename {\tt rbinomial.py}.

\section{Discussion}

\step First, import your uniform random number generator library and set the seed of the random number generator.
(Otherwise you will always get the same Bernoulli trials.) 

\begin{pyverbatim}
from varunif import runif
from varunif import setseed

_x = _seed = int(round(time.time() * 1000)) 
# or, I defined a function in runif.py to do that to hide implementation details
setseed( int(round(time.time() * 1000)) )
\end{pyverbatim}

In this case, we're using the current time in milliseconds as the random seed so that it is different every time you run the program. (remember this trick.)

\step Next, define a function that performs $n$ Bernoulli trials with probability $p$ of success. It should return the number of successes out of $n$:

\begin{pyverbatim}
def binomial(n,p):
    "Sim with prob p, n bernoulli trials; return number of successes"
    ...
\end{pyverbatim}

The pseudocode is just a loop that goes around $n$ times and uses a variable from $U(0,1)$, using your {\tt runif()} function, to check for success or failure. For example, my solution assumes failure if the uniform random variable is greater than $p$.
    
\step Now that we can know how to get a binomial random variable, we can examine the binomial distribution.  All we have to do is grab a vector of, say, $SAMPLES$ binomial random values and the plot a histogram.  The density function at $k$ is just how many successes out of $SAMPLES$ there were ($k/SAMPLES$, actually).

Let me introduce you to something called a {\em list comprehension} in Python, which is a for loop that results in a list. It's also considered a {\em map} function ala {\em map-reduce}.  Get list $X$ as $SAMPLES$ binomial values with parameters $n=500$ and $p=0.4$. Do that by simply calling the {\tt binomial} function $N$ times.

\begin{pyverbatim}
X = [binomial(n,p) for t in range(SAMPLES)]
\end{pyverbatim}

\step Plot the histogram normalized (normed=1) and run it. (You'll need {\tt hist()}.) You should see a graph similar to the following:

\scalebox{.35}{\includegraphics{figures/rbinomial-5000-4.pdf}}

\step We could use the built-in binomial mass function but let's define our own since it's easy:

\[\tag{Binomial mass function}
\binom{n}{k} p^k (1-p)^{n-k}
\]

\noindent That's the probability that there are $k$ successes in $n$ trials with probability $p$ of success. Define a function like this:

\begin{pyverbatim}
def binom(n, k, p):
    """
    If we run n trials with p prob for each trial of success,
    how many have k successes?
    """
    ...
\end{pyverbatim}

\noindent You may use function {\tt scipy.misc.comb()} to compute $n \choose k$, but otherwise do the arithmetic yourself. (There is no loop in this function.)

\step To show the real distribution on top, we need to iterate $k$ across the range $0..n$ used in our empirical  simulation above.  Since this is a mass function not a smooth density function, we can use every fifth value in the range. Let's also add some text to describe the parameters.

\begin{pyverbatim}
Y = [binom(n, k, p) for k in range(0,n+1,5)]
plt.bar(range(0,n+1,5), Y, color='red', align='center', width=1)
plt.axis([150,250,0,.05]) # set the axes so that we get a close-up
plt.text(160,0.04, '$n = %d$' % N, fontsize=16)
plt.text(160,0.037, '$p = %f$' % P, fontsize=16)
plt.text(160,0.034, '$SAMPLES = %d$' % SAMPLES, fontsize=16)
\end{pyverbatim}

In this case I am not using 0..1 for the axes coordinates of the text; the default is the values of the graph itself. sometimes this is useful.

\step Run it and you should see something like the following:

\scalebox{.40}{\includegraphics{figures/rbinomial-5000-4-fancy.pdf}}

Note that we use a bar chart for the binomial theoretical distribution and not a smooth graph because this is a mass function not a density function.

\section{Deliverables}

Please submit:

\begin{itemize}
\item your {\tt rbinomial.py} file with values $n=500, p=0.4, SAMPLES=5000$
\item submit a PDF of your final graph.
\end{itemize}

\end{fullwidth}

\chapter{Generating Exponential Random Variables}

\setcounter{problem}{1}
\section{Discussion}

\begin{fullwidth}

The goal of this lab is to generate random values from the exponential distribution using the {\em inverse transform method}.  You will show the histogram of the random values and then show the theoretical exponential distribution on top to verify your results. You will reuse your exponential distribution random variable generator for the central limit theorem lab later. Use filename {\tt stats/plot\_rexp.py} for the plotting code, but you will stick your exponential functions into the usual {\tt stats/stats.py} file.

\section{Steps}

\step First, create a function called {\tt rexp(lambduh)} in {\tt stats/stats.py} that returns a random value from the exponential distribution using the inverse transform method. To do that, you need the inverse {\em cumulative distribution function} (CDF) for the exponential distribution $Exp(\lambda)$. The {\em probability density function} for the exponential distribution is:

\[
p = F(x; \lambda) = \lambda e^{-\lambda x}
\]

\noindent Therefore the inverse function to get the $x$ value associated with a probability $p$, we use

\[
x = F^{-1}(p; \lambda) = -\frac{ln(1-p)}{\lambda}
\]

\noindent Your function should look like the following:

\begin{pyverbatim}
def rexp(lambduh): # lambduh mispelled to avoid clash with lambda in python
    # u = get value from U(0,1) then
    # return F^-1(u) for exp cdf F^-1
\end{pyverbatim}

\noindent Use your {\tt runif01()} function from previous labs.

\step To plot things, create file {\tt stats/plot\_rexp.py} and get a sample of exponential random variables into variable {\tt X} of size N from $Exp(1.5)$ using your {\tt rexp()}. I usually define constants to make the code more readable:

\begin{pyverbatim}
N = 1000
LAMBDUH = 1.5
\end{pyverbatim}

\noindent then I can call {\tt rexp(LAMBDUH)} and change LAMBDUH everywhere in my code by just changing the constant. In this case, there's no real need but it's good practice.

\step  Plot a histogram of you are sample with {\tt bins=40, normed=1}.  You should see something like this:

\scalebox{.4}{\includegraphics{figures/exp-1_5-density.pdf}}

\noindent {\bf How do we know that this accurately represents the exponential distribution? We plot the theoretical distribution on top with a red line.}\\

\step Since it's easy, let's define our own exponential probability density function in {\tt stats/stats.py} as follows:

\begin{pyverbatim}
def exppdf(x, lambduh):
    ...
\end{pyverbatim}

\noindent When you call it, make sure use the same lambduh.

\step Now, before the call to function {\tt show()}, plot the theoretical distribution so that we can see both at once:

\scalebox{.4}{\includegraphics{figures/exp-1_5-density-fancy.pdf}}


\begin{callout}{\bcplume}
{\bf Deliverables}. Make sure that {\tt stats/stats.py} has the new functions and that {\tt stats/plot\_rexpr.py} is correctly committed to your repository and pushed to github. 
\end{callout}

\end{fullwidth}

\chapter{The Central Limit Theorem in Action}

\setcounter{problem}{1}

\section{Discussion}

\begin{fullwidth}

The goal of this lab is to observe how the sample means of uniform and exponential random variables have normal distributions with $N(\mu, \sigma^2/n)$ where $\sigma^2$ is the variance of the underlying distribution and $n$ is the sample size whose mean we compute. Use filenames {\tt clt\_unif.py}, {\tt clt\_exp.py} for this lab.

\section{CLT applied to uniform random variables}

\subsection{Steps}

\cut{
# The CLT in a nutshell says that the sample mean, X_, of samples
# from many distributions follows the normal distribution.
# Specifically, N(m, sigma^2/N) for sample size N.
# Here, we are using the exponential distribution with lambda=1.5
# and verifying that taking lots of both samples and getting
# a histogram shows a normal distribution of N(1/lambda, 1/lambda^2)
# where mean=1/lambda, variance=1/lambda^2 for exp distribution.
}

\step Import the usual libraries for plotting and then define these constants:

\begin{pyverbatim}
N = 4  # sample size (i.e, array size len(X))
TRIALS = 500 # how many samples we will take from the uniform distribution
\end{pyverbatim}

\noindent Now, we need to build a loop that gets  {\tt TRIALS} $X$ vectors of size $N$ with values from $U(0,1)$. Use your {\tt runif()} function. Compute the mean of each $X$ vector and add it to the end of a different array $X\_$.

\step Plot the histogram of {\tt X\_}:

\begin{pyverbatim}
# plot density of means (normalized histogram of means)
# WARNING: bins=40 is to show changes in resolution
#          where normally it's best to let the hist()
#          choose the bins for smoother view
plt.hist(X_, bins=40, normed=1)
\end{pyverbatim}

\step Your histogram should look like this \\

\scalebox{.35}{\includegraphics{figures/clt_unif-500-4-basic.pdf}}
    
Cool.  It looks kind of like a normal distribution to me. Let's add the theoretical normal distribution on top. To do that we need the appropriate parameters of $Normal(\mu, \sigma^2/n)$. The mean  $\mu$ of uniform samples should be midway between $a$ and $b$ from $U(a,b)$. In our case, that's 0.5 since we are doing $U(0,1)$. The variance of the uniform distribution is $(b-a)^2/12$ and we need the variance divided by sample size $N$.  Define a function that returns the variance of uniform distribution $U(a,b)$:

\begin{pyverbatim}
def unifvar(a, b):
    ...
\end{pyverbatim}

\step  To get the theoretical distribution, let's define it ourselves:

\begin{pyverbatim}
def normpdf(x, mu, sigma): # sigma is the standard deviation, sigma^2 is the variance
    ...
\end{pyverbatim}

\noindent The function in math notation is:

\[
P(x) = \frac{1}{{\sigma \sqrt {2\pi } }}e^{{{ - \left( {x - \mu } \right)^2 } \mathord{\left/ {\vphantom {{ - \left( {x - \mu } \right)^2 } {2\sigma ^2 }}} \right. \kern-\nulldelimiterspace} {2\sigma ^2 }}}
\]

\step Then, plot the theoretical normal distribution on top of the histogram and set the axes so that we can use the same range throughout the next series of tests to see how the distribution changes. Note that the usual normal density function provided above expects the {\bf standard deviation not the variance} and so we need to pass {\tt normpdf()} the square root of the expected variance.

\begin{pyverbatim}
# plot real normal curve N(0.5, sigma^2/n)
x = np.arange(min(X_), max(X_), 0.01)
y = stats.norm.pdf(x, 0.5, FILL THIS IN))
plt.axis([.1,.9,0,7])
plt.plot(x,y, color='red')
\end{pyverbatim}

\step Run it. The resulting graph should look like this \\

\scalebox{.35}{\includegraphics{figures/clt_unif-500-4-fancy.pdf}}

\step Now, display some important parameters in the graph using {\tt text()}. You will need to do that {\tt fig.add\_subplot(111)} thing again early in your script. The text in between the \$ symbols is latex and lets us display nice math symbols (e.g., the title), although I'm not doing much with it here.

{\small
\begin{pyverbatim}
plt.text(.02,.9, '$N = %d$' % N, transform = ax.transAxes)
plt.text(.02,.85,'$TRIALS = %d$' % TRIALS, transform = ax.transAxes)
plt.text(.02,.8, 'mean($\\overline{X}$) = %f' % np.mean(X_), transform = ax.transAxes)
plt.text(.02,.75,'var($\\overline{X}$) = %f' % np.var(X_), transform = ax.transAxes)
plt.text(.02,.7, 'var U($0,1$)/%d = %f' % (N,varunif(0,1)/N), transform = ax.transAxes)

plt.title("CLT Density Demo. sample mean of U(0,1) is $N(.5, \sigma^2/N)$")
plt.xlabel("$\\overline{X}$", fontsize=16)
plt.ylabel("Density", fontsize=16)
\end{pyverbatim}
}

\step Run it. The resulting graph should look like this \\

\scalebox{.35}{\includegraphics{figures/clt_unif-500-4-fancier.pdf}}

Notice how the mean is close to the expected 0.5 and that the variance of the sample mean is close to the theoretical variance.\\

\step Increasing the number of trials two 2000 shows much higher resolution but does not change the variance/tightness of the distribution at all. Run it and see the following:

\scalebox{.35}{\includegraphics{figures/clt_unif-2000-4.pdf}}

\step Now, if we increase the sample size to $N=10$, we get a much tighter variance on the mean of $\overline{X}$. Run it:

\scalebox{.35}{\includegraphics{figures/clt_unif-2000-10.pdf}}

\step Increasing to 20 we get:

\scalebox{.35}{\includegraphics{figures/clt_unif-2000-20.pdf}}

\section{CLT applied to exponential random variables}

Now let's look at how the central limit theorem still gives us a normal distribution even when we pull random variables from a skewed distribution like the exponential. Create and edit a new file {\tt clt\_exp.py}.

\subsection{Steps}

\step Import the {\tt rexp(lambduh)} function you wrote for the previous lab to get exponential random variables and start out with the following constants:

\begin{pyverbatim}
N = 10
TRIALS = 4000
LAMBDUH = 1.5
\end{pyverbatim}

\step Repeat the loop we did above to get the mean of a bunch of samples into $X\_$, but this time from the exponential distribution {\tt rexp(LAMBDUH)} instead of the uniform distribution function. Plot the histogram of $X\_$ as you did before.

\step Plot the theoretical normal distribution on top using your {\tt normpdf()} (you can cut/paste it into {\tt clt\_exp.py}). The mean of the exponential distribution is $\mu = \lambda^{-1}$ and its variance is $\sigma^2 = \lambda^{-2}$.

\begin{pyverbatim}
# plot real normal curve N(lambda^-1, sigma^-2 / N)
x = np.arange(min(X_), max(X_), 0.01)
y = normpdf(x, FILL IN MEAN, FILL IN STDDEV)
plt.plot(x,y, color='red')
\end{pyverbatim}

\step Here are the appropriate text annotations:

{\small
\begin{pyverbatim}
plt.text(.02,.9, '$N = %d$' % N, transform = ax.transAxes)
plt.text(.02,.85, '$TRIALS = %d$' % TRIALS, transform = ax.transAxes)
plt.text(.02,.8,   'mean($\\overline{X}$) = %f' % np.mean(X_), transform = ax.transAxes)
plt.text(.02,.75, 'var($\\overline{X}$) = %f' % np.var(X_), transform = ax.transAxes)
plt.text(.02,.7, 'mean Exp($%f$) = %f' % (LAMBDUH,1/LAMBDUH), transform = ax.transAxes)
plt.text(.02,.65, 'var Exp($%f$)/%d = %f' % (LAMBDUH,N,(1/LAMBDUH**2)/N), transform = ax.transAxes)

plt.title("CLT Density Demo. sample mean of Exp($\lambda=1.5$) is $N(1/\lambda, (1/\lambda^2)/N)$")
plt.xlabel("$\\overline{X}$", fontsize=16)
plt.ylabel("Density", fontsize=16)
plt.axis([0,1.333,0,5])
plt.savefig('clt_exp-'+str(TRIALS)+'-'+str(N)+'.pdf', format="pdf")
\end{pyverbatim}
}

\step Run it and you should see the following two graphs according to the value of $N$:

\noindent \scalebox{.33}{\includegraphics{figures/clt_exp-4000-10.pdf}} \scalebox{.33}{\includegraphics{figures/clt_exp-4000-50.pdf}}

Notice that there is a slight leftward bias in that the normal distribution is a little bit to the right it looks like. This is to be expected. You really need to bump up $N$ before you see it converge to the proper alignment.\\

\step Play around with other values of lambda and N.

\section{Deliverables}

Please submit:

\begin{itemize}
\item both {\tt clt\_unif.py}, {\tt clt\_exp.py} files
\item a PDF for $N=20$, $TRIALS=2000$ for CLT $U(0,1)$ demo
\item a PDF for $N=50$, $TRIALS=4000$, $\lambda=1.5$ for CLT $Exp(\lambda)$ demo
\end{itemize}

\end{fullwidth}

\chapter{Generating Normal Random Variables}

\setcounter{problem}{1}
\section{Discussion}

\begin{fullwidth}


The goal of this lab is to generate normal random variables but using the Central limit theorem instead of the inverse transform or the accept reject method. I'm not recommending this as the most efficient method, but it is a great practical application of the central limit theorem.  The hard part about all of this is using the right variance and shifting from $N(0,1)$ to the general $N(\mu, \sigma^2)$. Use filename {\tt rnorm.py}.

\section{Steps}

\step First, let's define some constants and the variance of a uniform variable (you should have this from the CLT lab already):

\begin{pyverbatim}
N = 100
TRIALS = 4000

def unifvar(a,b):
	return ((b-a)**2)/12.0
\end{pyverbatim}

\step To define a function that generates normal random variables in $N(0,1)$, we rely on the fact that the sample mean, $\overline X$ from a sample, $X$, of uniform distribution values is normal.  This gives us as many normal random values as we want, one per sample $X$. We just have to tweak things so that the mean of the distribution is zero-centered and has variance 1. That shifted and scaled value is what we return from {\tt rnorm01()}:

\begin{pyverbatim}
def rnorm01():
    "return a value from N(0,1)"
    ...
\end{pyverbatim}	

\noindent The process looks like this:

\renewcommand{\theenumi}{\Alph{enumi}}

\begin{enumerate}
\item Get $N$ uniform random values from $U(0,1)$ into $X$ using your {\tt runif()} function.
\item Then compute the mean $\overline X$.
\item Shift that value so that is zero-centered and call it $rv$.
\item We know from the CLT lab that the variance of random variable $\overline X$ is $\sigma^2 / N$, where $\sigma^2$ is the variance of the underlying distribution $U(0,1)$, but we need the variance to be 1. Scale $rv$ so that it has variance 1. Note that a ``standard normal'' variable can be created from an arbitrary normal $X$ via $Z = (X-\mu)/\sigma$. $Z$ is effectively a shifted and scaled version of the original.  Interestingly, it really just measures how many standard deviations $X$ is from $N(0,1)$.
\end{enumerate}

\step Now, let's fill in the code we need to draw a histogram and  the theoretical distribution on top using the {\tt normpdf()} from the CLT labs:

\begin{pyverbatim}
# Get X taken from TRIALS trials, plot histogram normalized to density func
X = [rnorm01() for i in range(TRIALS)]
plt.axis([-4, 7, 0, 0.5]) # let's keep the same access across plot for this lab
plt.hist(X, bins=40, normed=1) # histogram should look standard normal

# plot real normal curve
x = np.arange(min(X),max(X), 0.01)
MEAN = 0
VARIANCE = 1
y = normpdf(x, MEAN, math.sqrt(VARIANCE)) # recall our normpdf takes standard deviation as variance
plt.plot(x,y, color='red') 
plt.savefig('rnorm01-%d-%d.pdf' % (TRIALS,N), format="pdf")
plt.show()
\end{pyverbatim}

\scalebox{.45}{\includegraphics{figures/rnorm01-4000-100.pdf}}

\step Now define a more general method that accepts a desired mean and variance ({\em not the mean and the standard deviation}):

\begin{pyverbatim}
def rnorm(mean, variance):
    "return a value from N(mean,variance)"
    ...
\end{pyverbatim}

\noindent We know how to get a standard normal random variable, $Z$, as we just defined {\tt rnorm01()}. To get a normal random variable with different mean and variance, we reverse the process we used to get a standard normal via $Z = (X-\mu)/\sigma$. Dust off your high school algebra and solve for $X$. That tells you how to shift and scale properly: $X = \mu+ Z\sigma$.

\step And test as before but this time use $\mu=2$ and $\sigma^2 = 2$:

\begin{pyverbatim}
MEAN = 2.0
VARIANCE = 2.0
# Get X taken from TRIALS trials, plot histogram normalized to density func
X = [rnorm(MEAN,VARIANCE) for i in range(TRIALS)]
plt.hist(X, bins=40, normed=1) # histogram should look gaussian

# plot real normal curve
x = np.arange(min(X),max(X), 0.01)
y = normpdf(x, MEAN, math.sqrt(VARIANCE))
plt.plot(x,y, color='red') 
plt.savefig('rnorm-%d-%d-%d-%d.pdf' % (MEAN,VARIANCE,TRIALS,N), format="pdf")
plt.show()
\end{pyverbatim}

\noindent You should get the following graph:

\scalebox{.45}{\includegraphics{figures/rnorm-2-2-4000-100.pdf}}

\section{Deliverables}

Please submit:

\begin{itemize}
\item your {\tt rnorm.py} file and please use the usual "if main" gate so that I can import your code for testing without creating the graphs:
\begin{pyverbatim}
if __name__ == '__main__':
\end{pyverbatim}

\item a PDF of the graphs shown above for $N(0,1)$ and $N(\mu=2,\sigma^2=2)$.
\item your {\tt varunif.py} used by your code

\end{itemize}

\end{fullwidth}


\chapter{Confidence Intervals for Price of Hostess Twinkies}

\setcounter{problem}{1}
\section{Goal}

\begin{fullwidth}


The goal of this lab is to learn how to compute an empirical 95\% confidence interval for sample means using an awesome technique called {\em bootstrapping}. As part of this lab, you will also learn to read in a file full of numbers. In this case, we are going to read in the price of Hostess Twinkies, a tasty snack recently returned from the dead, from around the US.

\section{Discussion}

A sample mean confidence interval of 95\% tells us the range in which most (95\% or $1.96\sigma$) of the sample means fall.  All we have to do is create a number of samples, $X$, and compute the means $\overline{X}$.  If we do this lots of times (trials) then 95\% of the time, we would expect the sample mean to fall within the range of 95\% of the samples. We just have to order the $\overline{X}$ values and strip off the lower and top 2.5\%. Then, the lowest and highest value in that stripped list represent the boundaries of the confidence interval. Cool, right?

From the central limit theorem, we know that the distribution of $\overline{X}$ is $N(\overline{X}, \sigma^2/n)$ for sample size $n$ (not the number of trials). In this case, though we don't know what the underlying distribution is because we just got a bunch of prices from a file. We could assume that it's normally distributed, but there's no point. The central limit theorem works on any underlying distribution we care about here but we do need the variance. For that, we can use the sample variance as an estimate of the variation in the overall Hostess Twinkies price population.

The question is how do we get lots of trials from an underlying distribution that we cannot identify? By repeated sampling from our single sample {\em with replacement}. This is called {\em bootstrapping}, which you could also call {\em resampling}. The idea is to randomly select $N$ values from our known data set of size $N$. That gives us one trial. We can then repeatedly compute our test statistic, the mean, on each sample.

To verify that we are doing the right thing, we will draw the theoretical normal distribution expected by the Central limit theorem and then shade in the 95\% theoretical confidence interval, which we know is 1.96 standard deviations on either side of the mean: $\mu \pm 1.96\sigma$.

Please do your work in filename {\tt conf.py}.

\section{Steps}

\step First, we have to get the data into a file called {\tt prices.txt}:

\begin{pyverbatim}
prices = []
f = open("prices.txt")
for line in f:
	v = float(line.strip())
	prices.append(v)
\end{pyverbatim}

When debugging or during development, you can print those numbers out to verify they look okay.

\step Now, we need a function that lets us sample {\em with replacement} from that raw data set. In other words, we need a function that gets $n$ values at random from a data parameter (a list of numbers). It should allow repeated grabbing of the same value (that's what we call with replacement).

\begin{pyverbatim}
def sample(data):
	"""
	Return a random sample of data values with replacement.
	The returned array has same length as data.
	"""
\end{pyverbatim}

The idea is to get an array of random numbers from $U(0,n)$ for {\tt n=len(data)}. These then are a set of indices into the data array so just loop through this index array grabbing values from data according to the index. For example if you have indexes = [3,9] for a 2-element data array, then return a new array [data[3], data[9]. My solution has two lines in it.

\step Now define TRIALS=20 and perform that many samplings of prices. For each sample, create the sample mean and add it to an X\_ list.

\step Sort that list and get the values from TRIALS*0.025..TRIALS*0.975 in X\_ and call it {\tt inside}.

\step Print the first and last value of the inside array as that will tell you what the bounds of your 95\% confidence interval are

\begin{pyverbatim}
print inside[0], inside[-1]
\end{pyverbatim}

\noindent You might get something like (there will be a lot of variation):

1.12295362319 1.16113333333

\step Add code to plot diamonds on the graph at those locations:

\begin{pyverbatim}
plt.plot(inside[0], 0, 'bD')
plt.plot(inside[-1], 0, 'bD')
\end{pyverbatim}

\step Now plot the normal curve using your amazing new understanding of the central limit  theorem. Use the following range and also set the overall graph range:

\begin{pyverbatim}
x = np.arange(1.05, 1.25, 0.001)
plt.axis([1.10, 1.201, 0, 30])
\end{pyverbatim}

\step Run it and you should get the following graph:

\scalebox{.4}{\includegraphics{figures/conf-20-basic.pdf}}

Ok, that's great but we have no idea if this is correct or not. Now, let's go nuts and show lots of stuff on the graph.

\step First, let's shade in the theoretical 95\% confidence interval using your {\tt normpdf()}.

\begin{pyverbatim}
mean = ...
stddev = ...
# redraw normal but only shade in 95% CI
left  = FILL IN
right = FILL IN

ci_x = np.arange(left, right, 0.001)
ci_y = normpdf(ci_x,mean,stddev)
# shade under (ci_x,ci_y) curve
plt.fill_between(ci_x,ci_y,color="#F8ECE0") 
\end{pyverbatim}

Run it again to see how it shades in the graph.

\scalebox{.4}{\includegraphics{figures/conf-20-basic2.pdf}}

\step Now let's annotate with lots of information. Please read through and figure out what all of that stuff does to draw the nice arrows and so on.

{\small
\begin{pyverbatim}
plt.text(.02,.95, '$TRIALS = %d$' % TRIALS, transform = ax.transAxes)
plt.text(.02,.9,  '$mean(prices)$ = %f' % np.mean(prices), transform = ax.transAxes)
plt.text(.02,.85, '$mean(\\overline{X})$ = %f' % np.mean(X_), transform = ax.transAxes)
plt.text(.02,.80, '$stddev(\\overline{X})$ = %f' %
    np.std(X_), transform = ax.transAxes)
plt.text(.02,.75, '95%% CI = $%1.2f \\pm 1.96*%1.3f$' %
   (np.mean(X_),np.std(X_)), transform = ax.transAxes)
plt.text(.02,.70, '95%% CI = ($%1.2f,\\ %1.2f$)' %
				  (np.mean(X_)-1.96*np.std(X_),
				   np.mean(X_)+1.96*np.std(X_)),
		 transform = ax.transAxes)

plt.text(1.135, 11.5, "Expected", fontsize=16)
plt.text(1.135, 10, "95% CI $\\mu \\pm 1.96\\sigma$", fontsize=16)
plt.title("95% Confidence Intervals: $\\mu \\pm 1.96\\sigma$", fontsize=16)

ax.annotate("Empirical 95% CI",
			 xy=(inside[0], .3),
			 xycoords="data",
			 xytext=(1.13,4), textcoords='data',
			 arrowprops=dict(arrowstyle="->",
                            connectionstyle="arc3"),
			 fontsize=16)
\end{pyverbatim}
}

\step Run it and you should get the following graph:

\scalebox{.4}{\includegraphics{figures/conf-20.pdf}}

\step  We don't have to increase the number of trials very much before the confidence interval tightens up nicely. Try 500:

\scalebox{.4}{\includegraphics{figures/conf-500.pdf}}

\section{Deliverables}

Please submit:

\begin{itemize}
\item your {\tt conf.py} file with TRIALS=500
\item a PDF of the graph with TRIALS=500 shown above.
\end{itemize}

\end{fullwidth}


\documentclass[titlepage]{tufte-book}

\usepackage[runall=true]{pythontex}
\setpythontexworkingdir{<outputdir>}
\usepackage{environ}
\usepackage{morewrites}
\usepackage{amsthm}
\usepackage{amsmath}
\usepackage{amssymb}
\usepackage[pdftex]{graphicx}
\usepackage{epstopdf}
\usepackage{hyperref}
\usepackage{alltt}
\usepackage{listings}
\usepackage{array}
\usepackage{extarrows}
\usepackage{setspace}
\usepackage{tikz}
\usepackage{tikz-qtree}
\usetikzlibrary{calc}
\usetikzlibrary{positioning}
\usepackage{hyperref}
\usepackage{graphviz}
\usepackage{geometry}                % See geometry.pdf to learn the layout options. There are lots.
\usepackage{bashful}
\usepackage{microtype} % Improves character and word spacing
\usepackage{caption}

\usepackage{booktabs} % Better horizontal rules in tables

\setkeys{Gin}{width=\linewidth,totalheight=\textheight,keepaspectratio} % Improves figure scaling
\graphicspath{{figures/}}

\usepackage{fancyvrb} % Allows customization of verbatim environments
\fvset{fontsize=\normalsize} % The font size of all verbatim text can be changed here

\newcounter{problem}
\newcounter{total}
\newcommand{\step}[1]{{}
\vspace{4pt} \noindent {\bf \theproblem. }#1\addtocounter{problem}{1}}

\newcommand{\cut}[1]{}

\usepackage[tikz]{bclogo}
\usepackage{tikz}
\usetikzlibrary{calc}

\lstdefinestyle{BashInputStyle}{
  language=bash,
  basicstyle=\small\ttfamily,
  numberstyle=\tiny,
  showstringspaces=false,
  numbersep=3pt,
  otherkeywords={|, ;, ', ", *,>, <, *, &, `, $},
  alsoletter={:~_},
  columns=fullflexible,
  backgroundcolor=\color{yellow!20},
  linewidth=0.93\linewidth,
  xleftmargin=0.03\linewidth,
  keywordstyle=\color{blue},
  emph={ls, cat, head, tail, more, less, sort, uniq, kill java, grep, zip, unzip, tar, wc, cp, chmod,chown},
  emphstyle=\color{black}\bfseries,
    commentstyle=\color{gray}\slshape
    }

\newcommand{\chili}{\scalebox{.04}{\includegraphics{figures/chili.pdf}}}
\newcommand{\chchili}{{\chili\chili}}
\newcommand{\chchchili}{{\chchili\chili}}

% The units package provides nice, non-stacked fractions and better spacing
% for units.
\usepackage{units}

% The fancyvrb package lets us customize the formatting of verbatim
% environments.  We use a slightly smaller font.
\usepackage{fancyvrb}
\fvset{fontsize=\normalsize}

% Small sections of multiple columns
\usepackage{multicol}

\hypersetup{
urlcolor=blue,
colorlinks=true
}
\usepackage[noline, procnumbered, linesnumberedhidden, boxed]{algorithm2e}

\newcommand{\openepigraph}[2]{ % This block sets up a command for printing an epigraph with 2 arguments - the quote and the author
\begin{fullwidth}
\sffamily\large
\begin{doublespace}
\noindent\allcaps{#1}\\ % The quote
\noindent\allcaps{#2} % The author
\end{doublespace}
\end{fullwidth}
}

\newcommand{\figref}[1]{Figure~\ref{#1}}
\renewcommand{\thefigure}{\arabic{figure}}

\newenvironment{callout}[1]{
\[
  \left[
      \begin{tabular}{@{\quad}m{.05\textwidth}@{\qquad}m{.75\textwidth}@{\quad}}
        \scalebox{1.5}{#1} & 
          \raggedright%
}
{
      \end{tabular}
    \right]
\]
}

\newcommand{\blankpage}{\newpage\hbox{}\thispagestyle{empty}\newpage} % Command to insert a blank page

\usepackage{makeidx} % Used to generate the index
\makeindex % Generate the index which is printed at the end of the document

\renewcommand{\maketitlepage}[0]{%
  \cleardoublepage%
  {%
  \sffamily%
  \begin{fullwidth}%
  ~
  \vspace{11.5pc}%
  \fontsize{36}{40}\selectfont\par\noindent\textcolor{darkgray}{\allcaps{\thanklesstitle}}%
  
\scalebox{.2}{\includegraphics{figures/msan-logo}}
  \vspace{11.5pc}%
  \fontsize{12}{18}\selectfont\par\indent\textcolor{darkgray}{\allcaps{\thanklessauthor}\\
\indent{\tt parrt@cs.usfca.edu}\\
\href{http://parrt.cs.usfca.edu}{http://parrt.cs.usfca.edu}}%
  \vspace{11.5pc}%
  \fontsize{14}{16}\selectfont\par\noindent\allcaps{\thanklesspublisher}%
  \end{fullwidth}%
  }
  \thispagestyle{empty}%
  \clearpage%
}

\titlecontents{part}% FIXME
    [0em] % distance from left margin
    {\vspace{1.5\baselineskip}\begin{fullwidth}\LARGE\rmfamily\itshape} % above (global formatting of entry)
    {\contentslabel{2em}} % before w/label (label = ``II'')
    {} % before w/o label
    {\rmfamily\upshape\qquad\thecontentspage} % filler + page (leaders and page num)
    [\end{fullwidth}] % after

  \titlecontents{chapter}%
    [0em] % distance from left margin
    {\vspace{1.5\baselineskip}\begin{fullwidth}\Large\rmfamily\itshape} % above (global formatting of entry)
    {\hspace*{0em}\contentslabel{2em}} % before w/label (label = ``2'')
    {\hspace*{4em}} % before w/o label
    {\rmfamily\upshape\qquad\thecontentspage} % filler + page (leaders and page num)
    [\end{fullwidth}] % after

\titlespacing*{\chapter}{0pt}{0pt}{30pt}
\titlespacing*{\section}{0pt}{3.5ex plus 1ex minus .2ex}{2.3ex plus .2ex}
\titlespacing*{\subsection}{0pt}{3.25ex plus 1ex minus .2ex}{1.5ex plus.2ex}

\begin{document}

\chapter{Is Free Beer Good For Tips?}

\setcounter{problem}{1}
\section{Goal}

\begin{fullwidth}

The goal of this project is to test a hypothesis using a variety of techniques: ``eyeball'' test, t-test, and bootstrapping.  Hypothesis testing with $p$-values is the inverse problem of confidence intervals.  Hypothesis testing examines the area outside the, say, 95\% interval; i.e., the $\alpha$=5\% significance level.

\section{Discussion}

Here is a typical statistics question (derived from one by Jeff "The Hammer" Hamrick) that we will solve in multiple ways.\\

{\bf Q.} {\em Psychologists studied the size of the tips in a restaurant on a given day when the waitron gave the patron a free beer. Here are tips from 20 patrons, measured in percent of the total bill: 20.8, 18.7, 19.1, 20.6, 21.9, 20.4, 22.8,
        21.9, 21.2, 20.3, 21.9, 18.3, 21.0, 20.3,
        19.2, 20.2, 21.1, 22.1, 21.0, and 21.7. Does a beer-inspired tip exceed 20 percent or perhaps dip below 20 percent (maybe patrons get drunk and can't do math)? Use a significance level equal to $\alpha$ = 0.06.}\\
        
\begin{callout}{\bctakecare}
Always pick the $\alpha$ significance level before you run your experiment. It is really bad mojo to pick your significance after you know what the p-value is.
\end{callout}

Before starting on this exercise, let's interpret the question. It asks whether the mean of the given experimental sample differs significantly from the usual 20\% tip. By ``significantly'' we mean ``statistically distinguishable'' not ``a lot.'' By ``usual'' we mean our {\em control} situation in which customers tip according to a normal distribution distribution centered at $\mu=20$ with variance $\sigma^2$. Formally,\\
~\\
$H_0: m = 20.0$ (non-free beer situation)\\
$H_1: m \neq 20.0$ (free beer situation; two-sided alternative hypothesis)\\
~\\
\noindent We could also say that $H_0: m - \mu = 0$ and $H_1: |m-\mu| > 0$.

On a typical day before running this experiment, we would expect tips to bounce  to the left and right of the population mean of 20.0 with some variance $\sigma^2$. The average on a given day would therefore follow distribution $N(20.0, \sigma^2/n)$ for, let's say, a fixed $n$ customers per day. The question is, does this particular  experiment's sample mean, $m=20.725$, fall outside of the typical variability of the sample means? {\bf Note that we are comparing sample means not tips}. The {\bf null hypothesis} is that the mean for the specified sample does not differ significantly from $\mu = 20.0$.   The {\bf alternate hypothesis} is that the sample mean differs significantly above or below the population mean.  

\section{Steps}

\subsection{Eyeballing it}

\step First, just draw a histogram of the tips to see what it looks like. I used {\tt plt.hist(tips, bins=5, normed=1)}. For this exercise, create a file called {\em userid}{\tt -hyp/hist.py}.

\scalebox{.4}{\includegraphics{figures/tips-histo.pdf}}

\noindent For your convenience, here are the tips in python format:

\begin{pyverbatim}
tips = [20.8, 18.7, 19.1, 20.6, 21.9, 20.4, 22.8,
        21.9, 21.2, 20.3, 21.9, 18.3, 21.0, 20.3,
        19.2, 20.2, 21.1, 22.1, 21.0, 21.7]
\end{pyverbatim}

\noindent To me, there is a lot of ``mass'' to the right of the usual 20\% tip but my eyeball is not a rigorous significance mechanism. 

\step To get a better idea, let's plot the distribution of the {\em sample means} given our $H_0$ assumption: $N(20.0, s^2/n)$.  (We can use the sample variance $s^2$ from our experimental sample for $\sigma^2$ because we don't know the variance of the original distribution. It safe to assume that the variance is similar.)  This is our ``control'' or the usual tipping distribution: the distribution of the average tips per day if $H_0$, the control, is true. Please use file {\em userid}{\tt -hyp/sample-mean-dist.py}.  

\scalebox{.45}{\includegraphics{figures/tips-means-dist.pdf}}

Looking at that graph, it seems that a sample mean of 20.73 is pretty far in the right tail of a normal curve centered at the control average 20\% tip. It looks to be a few standard deviations away from the mean. My gut says that it's pretty likely that giving people a free beer increases tips significantly.

\subsection{t-test}

\setcounter{problem}{1}

\step Let's use a {\em t-test} now to test for significance, just like we would do in statistics class. The $t$ value measures the number of standard deviations a sample mean, $m$, is away from our presumed population mean $\mu$: \\

\[\tag{t-value}
t = \frac{m - \mu}{s / \sqrt{N}}
\]

\noindent It's just the difference between the means scaled to be in units of standard deviations.  Write some code to compute the $t$-value. When computing $s$, the sample standard deviation, note that the numpy {\tt std()} function returns a biased estimate of the standard deviation. Use {\tt np.std(tips, ddof=1)} instead of just {\tt std(tips)}.  Please create file {\em userid}{\tt -hyp/t-test.py}.

\step Print out the value of $t$.  I get $t = 2.69417199392$. That means that $m$ is about 2.7 standard deviations away from $\mu$, which is a very significant departure. 

\step  To get a $p$-value, the likelihood that we would see such a $t$ value in the nonfree beer situation, look up $t$ in a $t$-distribution c.d.f. using {\tt 1-scipy.stats.t.cdf(t,N-1)}. (You might need to install scipy.) You should get $0.0071844$. Since we need to check both tails, the probability is actually $2 \times$ that, or, $p$-value=$0.014369$ (1.4\%). The definition of significance is $\alpha = 0.06$, which means that our sample mean is definitely a significant departure from the control mean 20.0 since $1.4\% < 6\%$.  There is only a 1.4\% chance that the control could generate a value that extreme or beyond. Here is my program output:

\begin{alltt}
t is 2.6941720
one-sided p-value is 0.0071844
two-sided p-value is 0.0143687
p-value (from t-test) = 0.014369, Reject H0
\end{alltt}

We must conclude that $m$ differs significantly from $\mu = 20.0$ based upon the significance of $\alpha=0.06$ and, therefore, we reject $H_0$ in favor of $H_1$.  Giving out free beers is likely to have increased the average tip in that experiment. Again, the $p$-value doesn't say anything about the magnitude of the difference, only that they are statistically distinguishable. An average tip of 20.7 may not be a huge increase but giving free beer does increase tip size.

\subsection{Boostrapping for empirical hypothesis testing}

\setcounter{problem}{1}

Ok, now, let's use bootstrapping to estimate a {\em p-value}.  Just to hammer it home, a $p$-value for some point statistic is the probability that the control (null hypothesis $H_0$) could generate that statistic. In our case, a $p$-value can tell us the likelihood that tips drawing from a normal distribution centered around $\mu=20.0$ with $s^2=var(tips)$ could result in a daily sample mean of 20.725. We are sampling from $N(\mu=20.0,s^2)$ to conjure up samples from the control situation. We are not resampling from the tips list as we are trying to see how the observed sample mean, 20.725, fits within the control distribution not the test distribution. {\em We are also not generating samples from the distribution of the sample mean random variable, $N(\mu=20.0,s^2/n)$}.

\begin{callout}{\bcinfo}
We know that the sample mean must follow distribution $N(20.0, \sigma^2/n)$, but there are point statistics where the central limit theorem does not apply. That motivates our examination of bootstrapping for empirical $p$-values.  We know that the central limit theorem applies to the sample mean and so we could go directly to the $N(20.0, \sigma^2/n)$ distribution in our simulation.  For other point statistics, we might need bootstrapping.
\end{callout}


\noindent Please use file {\em userid}{\tt -hyp/bootstrap.py}. You will also need a uniform random number generator. 

\step Bootstrap TRIALS=5000 samples of size $n=len(tips)$ from $N(\mu, s^2)$ using {\tt scipy.stats.norm.pdf()}. It's very important that we use the same sample size as $len(tips)$ so we are comparing the same thing. Compute the mean of each sample, $X$, and add to $\overline{X}$ as you generate samples from the normal distribution.

\step Compute how many means in $\overline{X}$ are greater than or equal to {\tt mean(tips)}:

\begin{pyverbatim}
greater = np.sum(X_ >= np.mean(tips))
\end{pyverbatim}

or

\begin{pyverbatim}
greater = sum([x>=np.mean(tips) for x in X_]) # the number of true values
\end{pyverbatim}

\step The (one-sided) p-value is just the ratio of values above the observed mean, mean(tips), to the number of trials. Double that because we're doing a two-sided test.  With 5000 trials, I see around 13 values greater than $m=20.725$. That gives us a p-value of $2*\frac{13}{5000} = 0.0052$ or .52\%. That means that, empirically, we find that there is an extremely small probability that the control could generate an extreme value like $m=20.725$. Certainly the likelihood is less than the required 6\% significant value.  Your output should look like:

\begin{alltt}
observed mean = 20.725
num greater than mean(tips) = 14
p-value from bootstrapping (ratio of X_ >= mean(tips)) is 0.0056
\end{alltt}

Note: we would expect the empirical p-value (.52\%) and the p-value derived from the t-test (1.4\%) to be very close to each other when the number of trials is large with bootstrapping.  Statistician Jeff Hamrick explains that the difference is not a problem with our bootstrapping solution and is ok.

``{\em A student t distribution with dof=19, is pretty close to a normal. But the differences are most greatly felt in the tails, and we're in the tails (rejection $H_{0}$), thus casting a little bit of sketchiness or your choice to draw the simulated raw data from a normal random variable. If we were performing this exact same operation on a data set with reasonably large size (say, 40 or 50 or 75) the differences would still exist but would be even more minute.}''

Again, we easily reject the control and conclude that giving out free beers increases tips.

\begin{callout}{\bcplume}
{\bf Deliverables}. {\em userid}{\tt -hyp/hist.py}, {\em userid}{\tt -hyp/sample-mean-dist.py}, {\em userid}{\tt -hyp/t-test.py}, and {\em userid}{\tt -hyp/bootstrap.py}.
\end{callout}

\end{fullwidth}

\end{document}

\part{Optimization and Prediction}
\chapter{Iterative Optimization Via Gradient Descent}

\section{Goal}

\begin{fullwidth}

The goal of this task is to increase your programming skill by solving an iterative computation problem with nontrivial iteration and termination conditions: {\em gradient descent function minimization}. Please use file {\tt opt/descent.py} for your {\tt minimize()} function and {\tt opt/plot\_descent.py} for the code that draws the trace of the minimization in action (i.e., this is the file that has all the matplotlib junk).

\section{Discussion}

Finding $x$ that minimizes function $f(x)$ (usually over some range) is an incredibly important operation as we use it to minimize risk and, for machine learning, to learn the parameters of our classifiers or predictors. Generally $x$ will be a vector but we will assume $x$ is a scalar to learn the basics. If we know that the function is convex like a quadratic polynomial, there is a unique solution and we can simply set the derivative equal to zero and solve for $x$:

\[\tag{Analytic solution to optimization}
f'(x) = 0
\]

\noindent For example, the function $f(x) = (x-2)^2 + 1$ has $f'(x) = 2x - 4$ whose zero is $x=2$.

\scalebox{.25}{\includegraphics{figures/quadratic.pdf}}

We prefer to find the {\em global minimum} but generally have to be satisfied with a {\em local minimum}, which we hope is close to the global minimum. A decent approach to finding the global minimum is to find a number of local minima via random starting $x_0$ and just choose the minimum local minimum discovered. For example, the function $f(x) = cos(3\pi x) / x$ has two minima in $[0,1.1]$, with one obvious global minimum:

\scalebox{.29}{\includegraphics{figures/cos-2minima.pdf}}
\scalebox{.29}{\includegraphics{figures/cos-2minima-edited.pdf}}

If the function has lots of minima/maxima or is very complicated, there may be no easy analytic solution.
There are many approaches to finding function minima iteratively (i.e., non-analytically), but we will use a well-known technique called {\em gradient descent} or {\em method of steepest descent}.  

\subsection{Gradient descent}

This technique can be used to train everything from {\em linear regression} models (see next lab) to {\em neural networks}.  Gradient descent requires a starting position, $x_0$, the function to optimize, $f(x)$, and its derivative $f'(x)$.  Recall that the derivative is just the slope of a function at a particular point. In other words, as $x$ shifts away from a specific position, does $y$ go up or down, and by how much?  E.g., the derivative of $x^2$ is $2x$, which gives us a positive slope when $x>0$ and a negative slope when $x<0$.  Gradient descent uses the derivative to iteratively pick a new value of $x$ that gets us closer and closer to the minimum of $f(x)$.   The negative of the derivative tells us the direction of the nearest minimum. For example, the graph to the right above shows a number of vectors representing derivatives at particular points. Note that the derivative is zero, i.e. flat, at the minima (same is true for maxima). The recurrence relation for updating our estimate of $x$ that minimizes $f(x)$ is then just:

\[
x_{i+1} = x_i - \eta f'(x_i)
\]

\noindent where $\eta$ is called the {\em learning rate}, which we'll discuss below. The $\eta f'(x_{i})$ term represents the size of the step we take towards the minimum. 
The basic algorithm is:

\begin{enumerate}
\item Pick an initial $x_0$, let $x = x_0$
\item Let $x_{i+1} = x_i - \eta f'(x_i)$ until $f'(x_i)=0$
\end{enumerate}

That algorithm is extremely simple but knowing when to stop the algorithm is problematic when dealing with the finite precision of computers. Specifically, no two floating-point numbers are ever equal really. So $f'(x) = 0$ is always false. Usually we do something like $abs(x_{i+1} - x_i) < precision$ or when $abs(f(x_{i+1}) - f(x_i)) < precision$ where precision is some very small number like 0.0000001.  Personally, I like the concept of stopping when there is a very small vertical change {\bf and} $f(x_{i+1})$ is heading back up.

The steps we take are scaled by the learning rate $\eta$.  Yaser S. Abu-Mostafa has some \href{http://www.amlbook.com/slides/iTunesU_Lecture09_May_01.pdf}{great slides} and videos that you should check out. Here is his description on slide 21 of how the learning rate can affect convergence:

\scalebox{.60}{\includegraphics{figures/stepsize.pdf}}

The domain of $x$ also affects the learning rate magnitude. This is all a very complicated finicky business and those experienced in the field tell me it's very much an art picking the learning rate, starting positions, precision, and so on. You can start out with a low learning rate and crank it up to see if you still converge without oscillating around the minimum.  \noindent An excellent description of gradient descent and other minimization techniques can be found in \href{http://apps.nrbook.com/fortran/index.html}{Numerical Recipes}.

\subsection{Approximating derivatives with finite differences}

Sometimes, the derivative is hard, expensive, or impossible to find analytically (symbolically).  For example, some functions are themselves iterative in nature or even simulations that must be optimized. There might be no closed form for $f(x)$. To get around this and to reduce the input requirements, we can approximate the derivative in the neighborhood of a particular $x$ value. That way we can optimize any reasonably well behaved function (left and right continuity would be nice). Our minimizer then only requires a starting location and $f(x)$ but not $f'(x)$, which makes the lives of our users much simpler and our minimizer much more flexible. 

To approximate the derivative, we can take several approaches. The simplest involves a comparison. Since we really just need a direction, all we have to do is compare the current $f(x_i)$ with values a small step, $h$, away in either direction: $f(x_{i}-h)$ and $f(x_{i}+h)$.  If $f(x_{i}-h) < f(x_{i})$, we should move $x_{i+1}$ to the left of $x_{i}$. If $f(x_{i}+h) < f(x_{i})$, we should move $x_{i+1}$ to the right.  This is called the forward difference but there is also backward difference and a central difference. The excellent article \href{http://research.microsoft.com/pubs/192769/tricks-2012.pdf}{\textcolor{blue}{Stochastic Gradient Descent Tricks}} has a lot of practical information on computing gradients etc...

Using the direction of the slope works, but does not converge very fast. What we really want is to use the magnitude of the slope to make the algorithm go fast where it's steep and slow where it's shallow because it will be approaching a minima. So, rather than just using the sign of the finite difference, we should use the magnitude or rate of change. Using finite differences then, we get a similar formula but replace the derivative with the finite (forward) difference:

\[
x _{i+1} = x_i - \eta \frac{f(x_{i}+h) - f(x_{i})}{h} \text{ where } f'(x) \approx \frac{f(x_{i}+h) - f(x_{i})}{h}
\]

\noindent To simplify things, we can roll the step size $h$ into the learning rate $\eta$ constant as we are going to pick that anyway.

\[
x _{i+1} = x_i - \eta (f(x_{i}+h) - f(x_{i}))
\]

\noindent  The step size is bigger when the slope is bigger and is smaller as we approach the minimum (since the region is flatter). Abu-Mostafa indicates in his slides that $\eta$ should increase with the slope whereas we are keeping it fixed and allowing the finite difference to increase the step size. We are not normalizing the derivative/difference to a unit vector like he does (see his slides).

\section{Your task}

You will use gradient descent to minimize $f(x) = cos(3\pi x) / x$. To increase chances of finding the global minimum, pick {\bf two} random locations in the range $[0.1,1.2]$ using standard python {\tt random.random()} and perform gradient descent with both of them. As part of your final submission, you must provide a plot of $f(x)$ with traces that indicate the steps taken by your gradient descent; use a different color for each descent. Here are two sample descents where the $x$ and $f(x)$ values are displayed as well as the minimum of those two:

\noindent \scalebox{.32}{\includegraphics{figures/cos-trace-2minima.pdf}}
\scalebox{.32}{\includegraphics{figures/cos-trace-2minima-left.pdf}}

To create the dots you just need to add the $x$ values to an array as you search for the minimum and then plot the $x$ and $f(x)$ values with red or green dots. In your {\tt opt/plot\_descent.py} file, you'll use stuff like:

\begin{pyverbatim}
tracey = [f(x) for x in tracex]
plt.plot(tracex, tracey, 'ro') # plot red dots
\end{pyverbatim}

Please show the information as I have shown in the graphs to make it easier to compare results and for me to grade.

Now, in your {\tt opt/descent.py} file, define a function called {\tt minimize} that takes the indicated parameters and returns a trace of all $x$ values visited including the initial guess:

\begin{pyverbatim}
def minimize(f, x0, eta, h, precision):
    tracex = []
    tracex.append(x0)  # add starting position
    ...
    return tracex
\end{pyverbatim}

\noindent Hide all of your plotting junk inside of {\tt opt/plot\_descent.py} file:

\begin{pyverbatim}
... code that uses minimize(), plotting ...
\end{pyverbatim}
    
As an example, I call that function like this:

\begin{pyverbatim}
tracex = minimize(f, x0, ETA, STEP, PRECISION)
\end{pyverbatim}

\noindent for an appropriate {\tt f()} definition per the above cosine function.  Note that Python allows us to pass a function just like any other object; we did this in our image {\tt filter()} function.  For parameter {\tt f}, we can call that function from within {\tt minimize()} with the usual syntax {\tt f()}.

So that we all have the same graph structure, please use the following code (in {\tt opt/plot\_descent.py}) to plot the cosine function:

\begin{pyverbatim}
import matplotlib.pyplot as plt

graphx = np.arange(.1,1.1,0.01)
graphy = f(graphx)
plt.plot(graphx,graphy)
plt.axis([0,1.1,-4,6])
\end{pyverbatim}

You will have to pick an appropriate step value $h$ to get a decent approximation of the derivative through finite differences but that is large enough to avoid faulty results from lack of precision (subtracting two floating-point numbers in the computer results in a number with much less precision than the original numbers). You want that number to be small enough so that your algorithm does not oscillate around the minimum. If the number is too big it will compute a finite difference that makes $x_{i+1}$ leap across the minimum to the other wall of the function. You must pick a learning rate $\eta$ that allows you to go as fast as you can but not so fast that it overruns the minimum back and forth. When I crank up my learning rate too far, I also see the algorithm oscillate:

\begin{pyverbatim}
...
f(0.491296576641) = -0.166774773584 , delta = 2.05763033375622805821
f(0.296744439739) = -3.171512867583 , delta = -3.00473809399913660556
f(0.297092626880) = -3.171512816769 , delta = 0.00000005081414267138
...
\end{pyverbatim}

To help you understand what your program is doing, print out $x$, $f(x)$, and any other value you think is helpful to see how your program explores the curve. BUT, your code shouldn't print that out in your final submission.

To give you some idea about  how fast your minimization function should converge my implementation seems to converge in less than 70 steps.

\section{Testing}

Test your code with the {\tt opt/test\_descent.py} program in the repo.
 
\begin{callout}{\bcplume}
{\bf Deliverables}.
\begin{itemize}
\item Your script {\tt opt/descent.py} with the {\tt minimize()} function.
\item Your script {\tt opt/plot\_descent.py}.
\item A PDF called {\tt opt/traces.pdf} of your graph with two {\em visible} traces (sometimes they will overlap and you can't see one of them).  It doesn't matter if they both are converging to the same minimum or two different ones. The graph should include the text I have on mine for $x$, $f(x)$, number of steps, etc...
\end{itemize}
Tag when completed with {\tt descent}.
\end{callout}

\end{fullwidth}


\input{regression-gradient-descent}

\part{Text Analysis}
\chapter{Summarizing Reuters Articles with TFIDF}

\section{Goal}

\begin{fullwidth}

The goal of this task is to learn a core technique used in text analysis called {\em TFIDF} or {\em term frequency, inverse document frequency}.  We will use what is called a {\em bag-of-words} representation where the order of words in a document don't matter---we care only about the words and how often they are present. A word's TFIDF value is often used as a feature for document clustering or classification. We will use it simply as a summarization tool for document. The more a term helps to distinguish its enclosing document from other documents, the higher its TFIDF score.

This task is also an opportunity learn how to organize Python code as a set of functions rather than an unstructured script (blob) with a bunch of global variables. You will also learn how to translate some simple algorithms written in pseudocode to Python code. As a practical matter, you will learn how to process XML files in Python.

\section{Discussion}

One way to summarize a text document is to list, say, the top 25 words that seem most important. That could also be used to compare documents to see if they're talking about the same thing. For example, I had to solve a problem 15 years ago to reduce noise in the forums of a Java developer's website.  Users were posting stupid posts about movies and were also putting database questions in the forum on GUIs. The goal was to detect non-Java posts and also to detect misplaced posts. What does it mean to ``talk about Java''?  How do I know when someone is talking about databases versus GUIs? My solution was to identify the words important to Java as a whole (``Java-speak''), database, and GUI posts.  Any posts that did not have words important to Java, were tossed out as irrelevant after giving them a mild smack on the snout. Similarly, posts without words relevant to databases were compared to vocabularies associated with other topics to see if another forum would be more appropriate. To make this work, I needed a precise definition of ``important words.'' As I did for that project, you will use a classic   text analysis technique called TFIDF in this project.

Certainly a word is important to a document if it's used a lot, but that would also include words like ``the'' so we need to discount words used frequently among our {\em corpus} (set of documents). So, we boost words used frequently in a document but attenuate if that word is used  in a lot of documents.  For more on this topic, see \href{http://nlp.stanford.edu/IR-book/html/htmledition/term-frequency-and-weighting-1.html}{Introduction to Information Retrieval}.

The {\em term frequency} is just the term count within a document divided by the number of words in that document (some people use ``frequency'' to mean ``count'' but that is an affront to the gods):

\[\tag{Term frequency of term $t$, document $d$}
tf(t,d) = \frac{count(t), t \in d}{|d|}
\]

A term's {\em document frequency} is the count of documents containing that term divided by the total number of documents:

\[\tag{Document frequency of $t$ in $N$ documents}
df(t,N) = \frac{|\{d_i : t \in d_i, \ i = 1..N\}|}{N}
\]

In order to reduce the TFIDF score for terms with high document frequencies, we need the document frequency  in the denominator:

\[\tag{First approximation to TFIDF}
\text{\em tfidf}(t,d,N) = \frac{tf(t,d)}{df(t,N)}
\]

This formula is  meaningful but gives a poor weight because the document frequency tends to overwhelm the term frequency fractions in the numerator so we take the log of the denominator first. The log also prevents division by 0 errors when a term does not exist in a corpus (e.g., in search applications where we pass random terms). Here's the formula slightly rewritten as it is normally shown:

\[\tag{TFIDF with attenuated document frequency}
\text{\em tfidf}(t,d,N) = tf(t,d) \times log(\frac{1}{df(t,N)})
\]

To summarize a document, we can order its terms by \text{\em tfidf} in reverse order and look at the top 20 words, for example.  To get the lexicon of a topic like databases, we can collect a known set of database posts into a single document and compute the tfidf in association with the aggregated documents of the other topics. Any word below a certain threshold, that we find by eyeballing it, is considered not relevant to that particular topic.

For example, here is a set of terms and the associated \text{\em tfidf} computed from a sample Reuters article. It's clear that it's talking about Nielsen ratings for news programs, without even looking at the original article.

\begin{table}[htdp]
\begin{center}
\begin{tabular}{|c|c|}
\hline
Term & \text{\em tfidf}\\
\hline
    rating     & 0.12332931962551781\\
    fox     & 0.11911646171233138\\
    nbc     & 0.11911646171233138\\
    homes     & 0.11408838230482544\\
    cbs     & 0.0794109744748876\\
    audiences     & 0.0794109744748876\\
    neilsen     & 0.0794109744748876\\
    evening     & 0.06678324701893232\\
    abc     & 0.06678324701893232\\
    watching     & 0.0634765565309808\\
\hline
\end{tabular}
\end{center}
\label{default}
\end{table}%

To compute TFIDF, we need an overall index that maps term $t$ to document frequency $df(t,N)$ for all $t$ in all $N$ documents and an index that maps document $d$ to another index that maps each $t \in d$ to $tf(t,d)$. From that, we can compute all of the TFIDF scores.  That is what you will do for this project, as described in the next section.

\section{Your task}

To implement this project, you have six key functions to implement. For the first four, you need to translate the following pseudocode to Python in a file called {\tt tfidf.py}.

\begin{function}[H]
\TitleOfAlgo{{\em words}(document $d$)}
\vspace{-4pt}
\KwIn{Document $d$}
\KwResult{non-unique list of words $wordlist$}
\Indp
 $wordlist$ = Split $d$ into words, removing numbers and punctuation\\
 Normalize $w \in wordlist$ to lowercase\\
 Strip out $w \in wordlist$ smaller than 3 letters\\
\Return{wordlist}
\end{function}

~\\

\begin{function}[H]
\TitleOfAlgo{{\em create\_indexes}(list of $files$)}
\vspace{-4pt}
\KwIn{List of filenames $files$}
\KwResult{Map document name to Counter object mapping term to frequency map {\em tf\_map}}
\KwResult{Counter object mapping term to document count $df$}
\Indp
$df$ = \{\}; {\em tf\_map} = \{\}\\
\ForEach{$f$ in files}{
 $d$ = $get\_text(f)$\\
 $words$ = $words(d)$\\
 $n$ = $len(words)$\\
 $tf$ = \{\}\\
 \lForEach{$t \in words$}{$tf[t]$ = $count(t) / n$}\\
 $tf\_map[f]$ = $tf$\\
 $df[t]$ += 1  ~~~~~\# not currently a frequency; it's a count
}
\Return{(tf\_map,df)}
\end{function}
~\\

\begin{function}[H]
\TitleOfAlgo{{\em doc\_tfidf}($tf, df, N$)}
\vspace{-4pt}
\KwIn{Term to frequency map $tf$}
\KwIn{Term to document count map $df$}
\KwIn{Number of documents $N$}
\KwResult{Map of each term in doc ($tf$) to TFIDF score}
\Indp
	$tfidf$ = \{\}\\
	\ForEach{$t \in tf$}{
		$df_t = df[t] / N$\\
		$idf_t = 1/df_t$ \\
		$tfidf[t] = tf[t] \times log(idf_t)$ \\
	}
	\Return{$tfidf$} \\
\end{function}

~\\


\begin{function}[H]
\TitleOfAlgo{{\em create\_tfidf\_map}($files$)}
\vspace{-4pt}
\KwIn{List of xml filenames $files$}
\KwResult{Map from file to map of term to TFIDF score}
\Indp
	$(tf\_map, df) = create\_indexes(files)$\\
	{\em tfidf\_map} = \{\}\\
	\ForEach{$f \in files$}{
		$tfidf$ = $doc_tfidf(tf\_map[f], df)$\\
		{\em tfidf\_map}$[f]$ = $tfidf$\\
	}
	\Return{tfidf\_map} \\
\end{function}

You must also provide a function called {\tt filelist(pathspec)} that returns a list of all files that match {\tt pathspec} and that {\em  have non-zero file sizes}:

\begin{pyverbatim}
def filelist(pathspec):
    ...
    return files
\end{pyverbatim}
	
For example, I might pass in {\tt\small ../data/reuters-vol1-disk1/*.xml}. Naturally, if this doesn't work, then the rest of the code will not work as it won't get the proper data.  That function should only consider the files in the specified directory, not subdirectories.  You will probably want to use Python function {\tt glob.glob()}.

To process XML files, you should also create the following function:

\begin{pyverbatim}
def get_text(fileName):
    """
    read an xml file and return the text from <title> and <text>
    """
    ...
    return text
\end{pyverbatim}

As part of your development work you will use lots of maps that look like \{dog: 36, cat: 19, ...\}.   Those integers, such as term counts, are easy to compute yourself but Python has an object that is effectively a histogram called \href{https://docs.python.org/2/library/collections.html#collections.Counter}{{\tt Counter}}. For example, if you give it a list of words, it will return an object that maps terms to their count. When you print them out, it will do so in reverse order of term count, which is very handy for testing.  Further, the unit tests I provide expect Counter objects.

For what it's worth, my implementation is just 60 lines including the {\tt import} statements. This is not a huge project but it is tricky when messing around with all of these maps of maps and lists of things. Start by understanding the problem and working a few TFIDF examples manually. Then, build a simple functions and test them individually before moving on to the more complex functions. For example, you should start by building {\tt filelist()} and then probably {\tt get\_text()}. My typical strategies to design from the top down and test from the bottom up.

\subsection{XML Input}

As part of this project, I will provide you with a set of Reuters news articles in XML format, which will be the input to your program. From it, you will create the appropriate indexes and I will test those values against what I computed with my solution.

The format of the files doesn't matter much except that you need to pull out the {\tt title} and {\tt text} tags. The {\tt p} paragraph tags inside {\tt text} need to be collected. All of this text is what you will return from {\tt get\_text()}.
 
\begin{alltt}
<?xml version="1.0" encoding="iso-8859-1" ?>
<newsitem itemid="131701" ...>
<title>German consumer confidence rises in Aug/Sept</title>
<text>
<p>German consumer confidence rose...</p>
<p>The Icon index, which...</p>
</text>
...
</newsitem>
\end{alltt}

{\bf The collection of Reuters articles is considered proprietary to Reuters and, to get access to the data, the faculty had to promise Reuters the data would not be made available on a public website or given to anyone else.} Please treat this data with care, do not posted to github, etc.

\subsection{Testing}

In computer science, programmers recognize two primary kinds of tests: {\em unit tests} and {\em functional tests}. A unit test is really just testing a function or a few functions whereas functional tests test the overall functionality of the program. In file {\tt test\_tfidf.py}, I have provided a set of unit and functional tests that you can use for basic sanity checking of your TFIDF project.  I would typically test your projects with a different set of unit tests but, in this case, we will define success as getting the correct answers for the large corpus of Reuters articles that I will provide to you.

To make the unit tests work, make sure that you install \href{http://pytest.org/latest/getting-started.html}{py.test}, which is usually just a matter of:

\begin{alltt}
easy_install -U pytest
\end{alltt}

I will test your code using the following command line (with your {\tt tfidf.py} is in the same directory):

\begin{alltt}
$ python -m pytest test_tfidf.py
============================= test session starts ==============================
platform darwin -- Python 2.7.7 -- py-1.4.20 -- pytest-2.5.2
collected 5 items 

test_tfidf.py .....

=========================== 5 passed in 0.04 seconds ===========================
\end{alltt}

\noindent If you don't see all tests passing, and there is a problem at a basic level with your software.

Note that the test file imports your file with:

\begin{pyverbatim}
from tfidf import *
\end{pyverbatim}

If you name it incorrectly, the program won't work. 

To test the entire corpus of Reuters articles, I will run your program as follows, potentially with a different path specification.
 
\begin{alltt}
python test_corpus.py '../data/reuters-vol1-disk1/*.xml'
\end{alltt}

\noindent Note that the quotations around the path specification are required to prevent the command line from expanding {\tt *.xml}. You want that path specification to go in as a single argument, not a list of files. The core of {\tt test\_corpus.py} is:

\begin{pyverbatim}
(file_to_histo, word_to_numdocs) = create_indexes(files)
for f in files:
    pairs = doc_tfidf(file_to_histo[f], word_to_numdocs, N)
    # convert map to a Counter object so we can use most_common()
    term_pair = Counter(tfmap).most_common(1)[0]
    print os.path.basename(f), "(\%s, \%1.4f)" \% (term_pair[0], term_pair[1])
\end{pyverbatim}

\noindent The output, which I have provided in file {\tt corpus\_output.txt.7z}, starts with:

\begin{alltt}
81880 files
131674newsML.xml (ewe, 0.3792)
131675newsML.xml (telefonica, 0.1701)
131676newsML.xml (ingenico, 0.1645)
131677newsML.xml (lisbon, 0.1397)
131678newsML.xml (warrant, 0.0490)
131679newsML.xml (billion, 0.1192)
13167newsML.xml (cents, 0.2657)
131680newsML.xml (tightness, 0.0628)
131681newsML.xml (telefonica, 0.1767)
131682newsML.xml (rate, 0.1994)
131683newsML.xml (asset, 0.0921)
131684newsML.xml (india, 0.0953)
131685newsML.xml (crowns, 0.0952)
131686newsML.xml (crowns, 0.1355)
131687newsML.xml (ireland, 0.1857)
...
\end{alltt}

My implementation takes about 1 minute 30 seconds to compute TFIDF scores for 81,880 XML files loaded from an SSD on a fast machine.  Loading those files takes just 5 seconds.
 
\section{Resources}

I provide for you the following files:

\begin{itemize}
\item {\tt test\_tfidf.py}: some simple tests using py.test.
\item {\tt test\_corpus.py}: prints out the file, term, and TFIDF score for the highest scoring term in each file.
\item {\tt corpus\_output.txt.7z}: compressed output from running test\_corpus.py
\item {\tt reuters-vol1-disk1-subset.7z}: compressed directory full of XML files---the corpus. It is 385M when uncompressed.
\end{itemize}

\section{Deliverables}

\noindent Please submit the following file via canvas:

\begin{itemize}
\item Your {\tt tfidf.py} file. {\em I will deduct a full point if your library is not executable exactly in the fashion mentioned in this project; that is, method names and filename must be exactly right. For you PC folks, note that case is significant for file names on unix! All projects must run properly under linux or OS X.}
\end{itemize}

\section{Extra credit --- Search Engine}

In this project, we created an index from term to the number of documents that contain that term. If we extend that to be an index from term to the list of documents containing the term, we can get the same results as we did before. The benefit would be that we could create a search engine.

Given a query such as ``consumer confidence,'' we could merge the list of files containing those two terms and display those to a user. It's fast and works great! The only problem is that we might get 1000 documents back and we'd really like to show the most relevant documents first.  Using the $tf\_map$ index, we can compute a relevance score for a query, relative to a document, by summing the TFIDF scores for each term in the query that is present in the document. The document with the highest two TFIDF scores for ``consumer confidence,'' would be the first document we displayed.

You need to modify the $df$ map from above and then implement the following function.

\begin{pyverbatim}
def search(query):
    docs = []
    # find list of documents for each term in query, put in docs
    # compute sum of TFIDF scores for each term in query relative to each document in docs
    # sort documents by reverse score
    return docs
\end{pyverbatim}

\end{fullwidth}



\cut{
Create Windows box

this tells you about the kinds of computers:
http://aws.amazon.com/ec2/instance-types/

 go to dashboard and select ec2

 click on launch instance

 use the classic Wizard then scroll down until you see a Windows machine, select "Microsoft Windows Server 2012 Base". 64bit.

select instance type "m1.small"

 skip over the advanced instance options

 skip over the storage device configuration

 for the key named "Name", change the value to something like <your user ID>-windows or something like that so that you can identify it later if you have multiple machines going.

 the first time you will need to create a new key pair. name it as your user ID. then click on the create and download keyfile. it will leave you with a parrt.pem file, which are your security credentials for getting into the machine. Save that file in a safe spot. If you lose it you will not be able to get back into your machine that you create. 

You can choose the default security group. this controls the firewall.

 we need port 3389 open! figure out how to do this. oh, go to security groups and alter the default. go to the inbounds tab and put 3389 into the port range and click add rule then apply.

tell it to create the instance. close that pop up and it will take you to the instances you. You will see that they will start to initialize and run the server.

right-click on the instance display and ask to retrieve the password. It will likely tell you that it is not yet available. In my experience it takes about 15 minutes before the server is up and you can get the passwd dialog to pop up with information. When you do, it will ask for your private key, which is that parrt.pem file that you saved earlier. click on choose file and located on your local disk. then click on decrypt password. cut-and-paste the decrypted password and put it into a text file to save it or wherever.

now click on connect to remote instance using a right-click. tell it to download the shortcut file. then just click on that file assuming you have installed Windows Remote Desktop. it will open up a dialog asking you to type in a password for the Administrator user. it might say something about this certificate is not correct. Don't worry about that and click okay to connect anyway. It will connect you to a window and show you that it is initializing a desktop for you on the remote server. Then you will see that you have control of the server via a desktop.
}


% -------------------- useful commands ----------------------------


\cut{
\newthought{Example of} the \texttt{newthought} command for starting new sections. Typography examples: \allcaps{all caps} and \smallcaps{small caps}.

\begin{marginfigure}
\includegraphics[width=\linewidth]{figures/tips-histo.pdf}
\caption{This is a margin figure. The helix is defined by $x = \cos(2\pi z)$, $y = \sin(2\pi z)$, and $z = [0, 2.7]$. The figure was drawn using Asymptote (\url{http://asymptote.sf.net/}).}
\label{fig:marginfig}
\end{marginfigure}

\marginnote{This is a random margin note. Notice that there isn't a number preceding the note, and there is no number in the main text where this note was written. Use \texttt{sidenote} to use a number.}

\begin{table} % Add the following just after the closing bracket on this line to specify a position for the table on the page: [h], [t], [b] or [p] - these mean: here, top, bottom and on a separate page, respectively
\centering % Centers the table on the page, comment out to left-justify
\begin{tabular}{l c c c c c} % The final bracket specifies the number of columns in the table along with left and right borders which are specified using vertical bars (|); each column can be left, right or center-justified using l, r or c. To specify a precise width, use p{width}, e.g. p{5cm}
\toprule % Top horizontal line
& \multicolumn{5}{c}{Growth Media} \\ % Amalgamating several columns into one cell is done using the \multicolumn command as seen on this line
\cmidrule(l){2-6} % Horizontal line spanning less than the full width of the table - you can add (r) or (l) just before the opening curly bracket to shorten the rule on the left or right side
Strain & 1 & 2 & 3 & 4 & 5\\ % Column names row
\midrule % In-table horizontal line
GDS1002 & 0.962 & 0.821 & 0.356 & 0.682 & 0.801\\ % Content row 1
NWN652 & 0.981 & 0.891 & 0.527 & 0.574 & 0.984\\ % Content row 2
PPD234 & 0.915 & 0.936 & 0.491 & 0.276 & 0.965\\ % Content row 3
JSB126 & 0.828 & 0.827 & 0.528 & 0.518 & 0.926\\ % Content row 4
JSB724 & 0.916 & 0.933 & 0.482 & 0.644 & 0.937\\ % Content row 5
\midrule % In-table horizontal line
\midrule % In-table horizontal line
Average Rate & 0.920 & 0.882 & 0.477 & 0.539 & 0.923\\ % Summary/total row
\bottomrule % Bottom horizontal line
\end{tabular}
\caption{Table caption text} % Table caption, can be commented out if no caption is required
\label{tab:template} % A label for referencing this table elsewhere, references are used in text as \ref{label}
\end{table}

}

\backmatter

\printindex % Print the index at the very end of the document



\end{document}