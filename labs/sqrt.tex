\chapter{Approximating $\sqrt{n}$ with the Babylonian Method}

\setcounter{problem}{1}
\section{Motivation}

\begin{fullwidth}

This lab shows how to encode and solve a recurrence relation from mathematics using iteration in Python to approximate a real-world, real valued function.  It also teaches how to quickly prototype something in Excel (if warranted).
 
\section{Discussion}

To approximate square root, $\sqrt{n}$, the idea is to pick an initial estimate, $x_0$, and then iterate with better and better estimates, $x_{i}$, using the recurrence relation:

\[
x_{i+1} = \frac{1}{2} (x_i + \frac{n}{x_i})
\]

To see how this works, jump into Excel (yes, a spreadsheet) and crank through a few iterations by defining cells with $n$ and your initial estimate $x_0$, which can be anything you want. (It's sometimes easier to play around without having to deal with a programming language.) Then you need to define a cell that computes the above better approximation using $x_i$ as the cell above it. I hardcoded the names in column A and the first two rows of column B. Cell B3 should be a formula that computes B4 based upon B3. Then you can extend the formula down and watch it converge on the final (correct) value for $\sqrt{125348}$. My spreadsheet looks like this:

~\\
\scalebox{.25}{\includegraphics{figures/sqrt-excel.pdf}}
~\\

Try out any nonnegative number and you'll see that it still converges, though at different rates.

There's a great deal on the web you can read to learn more about why this process works but it relies on the average (midpoint) of $x$ and $n/x$ getting us closer to $\sqrt{n}$.  It can be shown that if $x$ is above $\sqrt{n}$ then $n/x$ is below $\sqrt{n}$ and the reverse is true if $x$ is below the root.  The iteration converges and does so quickly. Informally, as shown in Wikipedia, we can represent the true square root by adding an error term to our estimate:

\[
\sqrt{n} = x + \epsilon
\]

or,

\[
n = (x + \epsilon)^2
\]

Expanding, we get:

\[
n = x^2 + 2x\epsilon + \epsilon^2
\]

Solving for $\epsilon$:

\[
n - x^2 = \epsilon (2x + \epsilon)
\]

\[
\epsilon = \frac{n - x^2}{2x + \epsilon}
\]

Because $\epsilon$ is much smaller than $x$, we can drop it from the denominator leaving us with an estimate of epsilon:

\[
\epsilon = \frac{n - x^2}{2x}
\]

Then we can plug it back into $x + \epsilon$ and get:

\[
x := x + \epsilon = x + \frac{n - x^2}{2x} = \frac{2x^2}{2x} + \frac{n - x^2}{2x} = \frac{1}{2}\frac{x^2 + n}{x} = \frac{1}{2}(x + \frac{n}{x})
\]

Which gets its back to the Babylonian formula. Since we dropped an $\epsilon$ term, this formula for $x$ is inexact but it gets us closer to $\sqrt{n}$.

\cut{
To play first use Excel: 
=0.5*(B2+\$B\$1/B2)
}

Now that you understand how this estimate works, your goal is to implement a simple Python method called sqrt() that uses the Babylonian method to approximate the square root. File {\tt sqrt.py} is in you are repository with some starter code. You will also find unit tests in {\tt stats/test\_sqrt.py}, which you can run with:

\begin{lstlisting}[style=BashInputStyle]
$ python -m pytest test_sqrt.py
\end{lstlisting}

\begin{callout}{\bctakecare}
You may not use math.sqrt() for implementing your function, but you may use it for testing the results.  Obviously.
\end{callout}

\begin{callout}{\bcplume}
{\bf Deliverables}. Make sure that {\tt stats/sqrt.py} is correctly committed to your repository and pushed to github. 
\end{callout}


\end{fullwidth}
