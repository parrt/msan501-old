\chapter{Generating Uniform Random Numbers}

\setcounter{problem}{1}

\begin{quote}
{\bf Q}: {\em How to generate pure random string}? \\
{\bf A}: {\em Put a fresh student in front of vi editor and ask him to quit}.\\
~~~~--- Emiliano Lourbet (@taitooz)
\end{quote}

\section{Discussion}

\begin{fullwidth}

To perform computer-based simulation we need to be able to generate random numbers. Generating random numbers following a uniform distribution are the easiest to generate and are what comes out of the standard programming language ``give me a random number'' function.  Here's a sample Python session:

\begin{pyconsole}
import random
print random.random()
print random.random()
print random.random()
\end{pyconsole}

We could generate real random numbers by accessing, for example, noise on the ethernet network device but that noise might not be uniformly distributed. We typically generate pseudorandom numbers that aren't really random but look like they are. From Ross' {\em Simulation} book,  we see a very easy recursive mechanism (recurrence relation) that generates values in $[0,m-1]$:\\

$x_n = a x_{n-1}$ modulo $m$\\

\noindent That is recursive (or iterative and not {\em closed form}) because $x_n$ is a function of a prior value: \\

$x_1 = ax_0$ modulo $m$\\
$x_2 = ax_1$ modulo $m$\\
$x_3 = ax_2$ modulo $m$\\
$x_4 = ax_3$ modulo $m$\\
$...$\\

\noindent To get random numbers between 0 and 1, we use $x_n / m$.

We must pick a value for $a$ and $m$ that make $x_n$ seem random. Ross suggests choosing a large prime number for $m$ that fits in our integer word size, e.g., $m = 2^{31} - 1$, and $a = 7^5 = 16807$.

Initially we set a value for $x_0$, called the {\em random seed} (it is not the first random number). Every seed leads to a different sequence of pseudorandom numbers. (In Python, you can set the seed of the standard library by using {\tt random.seed([x])}.)

Your goal is to take that simple recursive formula and use it to generate uniform random numbers. Please add the following functions file to the same {\tt stats/stats.py} file from previous exercises:

\begin{pyverbatim}
# U(0,1)
def runif():
    ...

# U(a,b)
def runif_(a,b):
    ...

def setseed(s): # updates the seed global variable
    ...
\end{pyverbatim}

Functions {\tt runif()} and {\tt runif\_()} return a new random value per call. Use $m = 2^{31} - 1$, $a = 7^5 = 16807$, and an initial seed of $x_0 = 666$.  {\em You may not use random.random() or any other built-in random number generators for this project.  Obviously.}

Use {\tt test\_runif.py} to test values.

\begin{callout}{\bcplume}
{\bf Deliverables}. Make sure that you update {\tt stats/stats.py} (the functions inside should emit no output at all, just return data as specified) and it is correctly committed to your repository and pushed to github. Tag when completed with {\tt runif}.
\end{callout}

\end{fullwidth}
