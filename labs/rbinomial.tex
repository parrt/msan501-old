\chapter{Generating Binomial Distributions}

\setcounter{problem}{1}
\section{Goal}

\begin{fullwidth}

The goal of this lab is to simulate a binomial distribution using repeated Bernoulli trials and then compare it against the theoretical binomial distribution. Use filename {\tt rbinomial.py}.

\section{Discussion}

\step First, import your uniform random number generator library and set the seed of the random number generator.
(Otherwise you will always get the same Bernoulli trials.) 

\begin{pyverbatim}
from varunif import runif
from varunif import setseed

_x = _seed = int(round(time.time() * 1000)) 
# or, I defined a function in runif.py to do that to hide implementation details
setseed( int(round(time.time() * 1000)) )
\end{pyverbatim}

In this case, we're using the current time in milliseconds as the random seed so that it is different every time you run the program. (remember this trick.)

\step Next, define a function that performs $n$ Bernoulli trials with probability $p$ of success. It should return the number of successes out of $n$:

\begin{pyverbatim}
def binomial(n,p):
    "Sim with prob p, n bernoulli trials; return number of successes"
    ...
\end{pyverbatim}

The pseudocode is just a loop that goes around $n$ times and uses a variable from $U(0,1)$, using your {\tt runif()} function, to check for success or failure. For example, my solution assumes failure if the uniform random variable is greater than $p$.
    
\step Now that we can know how to get a binomial random variable, we can examine the binomial distribution.  All we have to do is grab a vector of, say, $SAMPLES$ binomial random values and the plot a histogram.  The density function at $k$ is just how many successes out of $SAMPLES$ there were ($k/SAMPLES$, actually).

Let me introduce you to something called a {\em list comprehension} in Python, which is a for loop that results in a list. It's also considered a {\em map} function ala {\em map-reduce}.  Get list $X$ as $SAMPLES$ binomial values with parameters $n=500$ and $p=0.4$. Do that by simply calling the {\tt binomial} function $N$ times.

\begin{pyverbatim}
X = [binomial(n,p) for t in range(SAMPLES)]
\end{pyverbatim}

\step Plot the histogram normalized (normed=1) and run it. (You'll need {\tt hist()}.) You should see a graph similar to the following:

\scalebox{.35}{\includegraphics{figures/rbinomial-5000-4.pdf}}

\step We could use the built-in binomial mass function but let's define our own since it's easy:

\[\tag{Binomial mass function}
\binom{n}{k} p^k (1-p)^{n-k}
\]

\noindent That's the probability that there are $k$ successes in $n$ trials with probability $p$ of success. Define a function like this:

\begin{pyverbatim}
def binom(n, k, p):
    """
    If we run n trials with p prob for each trial of success,
    how many have k successes?
    """
    ...
\end{pyverbatim}

\noindent You may use function {\tt scipy.misc.comb()} to compute $n \choose k$, but otherwise do the arithmetic yourself. (There is no loop in this function.)

\step To show the real distribution on top, we need to iterate $k$ across the range $0..n$ used in our empirical  simulation above.  Since this is a mass function not a smooth density function, we can use every fifth value in the range. Let's also add some text to describe the parameters.

\begin{pyverbatim}
Y = [binom(n, k, p) for k in range(0,n+1,5)]
plt.bar(range(0,n+1,5), Y, color='red', align='center', width=1)
plt.axis([150,250,0,.05]) # set the axes so that we get a close-up
plt.text(160,0.04, '$n = %d$' % N, fontsize=16)
plt.text(160,0.037, '$p = %f$' % P, fontsize=16)
plt.text(160,0.034, '$SAMPLES = %d$' % SAMPLES, fontsize=16)
\end{pyverbatim}

In this case I am not using 0..1 for the axes coordinates of the text; the default is the values of the graph itself. sometimes this is useful.

\step Run it and you should see something like the following:

\scalebox{.40}{\includegraphics{figures/rbinomial-5000-4-fancy.pdf}}

Note that we use a bar chart for the binomial theoretical distribution and not a smooth graph because this is a mass function not a density function.

\section{Deliverables}

Please submit:

\begin{itemize}
\item your {\tt rbinomial.py} file with values $n=500, p=0.4, SAMPLES=5000$
\item submit a PDF of your final graph.
\end{itemize}

\end{fullwidth}
